% BACHELORARBEIT PRÄAMBEL
% Autor..................................Neele Elbersgerd
% Erstellungsdatum.......................10.08.2020

\documentclass[
fontsize=10pt, 
paper=a4, 
oneside,
%fleqn,  									% abgesetzte Formeln werden linksbündig ausgerichtet
%parskip=half+, 							% halber Zeilenabstand nach Absatz, +: min. ein Drittel der letzten Zeile eines Absatzes leer>> laut APA nicht nötig, daher raus
%headsepline = true,						%Linie unter Kopfzeile
headheight=30pt,
footheight=17pt,
abstract = true, 							% Überschrift im Abstract
%pointednumbers, 							% Punkt nach Gliederungszahl
toc=listof, toc=bibnumbered, 				% was soll in table of contents
listof=nochaptergap,
%toc=chapterentrywithdots,					% im toc auch Punkte zwischen Chapter und Seitenzahl
headings=optiontohead,						% optionale Befehle für Kolumnentitel
%noapacite,
numbers= noendperiod,
BCOR=8mm									% Bindekorrektur links --> wie viel?
]{scrreprt}
\AfterTOCHead{\linespread{1.25}\selectfont}	% Verzeichnisse mit kleinerem Zeilenabstand

% Allgemeines
		\usepackage[automark]{scrlayer-scrpage} 	% Kopf- und Fußzeilen
		\usepackage{amsmath,marvosym, amssymb} 		% Mathesachen
		\usepackage[TS1, T1]{fontenc} 				% Ligaturen, richtige Umlaute im PDF
		\usepackage[utf8]{inputenc}					% UTF8-Kodierung für Umlaute usw
		\usepackage[ngerman]{babel} 				% Silbentrennung
		\usepackage{mathtools} 						% mathematische Symbole
		\usepackage{upgreek}						% aufrechte griechische Buchstaben
		\usepackage[right]{eurosym}					% Eurosymbol
		\usepackage{siunitx}						% Einheiten
	%wie soll LaTeX bestimmte Wörter trennen?
		\hyphenation{Stab-elek-tro-de}
		\hyphenation{Haut-leit-wert-re-ak-tion}

% Absätze, Abstände, Seitenformat
		\usepackage[a4paper, left = 26mm, right = 26mm, top = 26mm, bottom = 26mm, 
		%showframe									% zeigt Satzspiegelränder
		]{geometry}    								% Textbreite
		%\setlength{\headsep}{0.7cm}				% Abstand zwischen Kopfzeile & Rumpf
		%\setlength{\footskip}{0.7cm}				% Abstand zwischen Fußzeile & Rumpf
		\usepackage{setspace} 						% 1.5 Zeilenabstand wie es üblicherweise 
			\setstretch{1.4}							% verstanden wird
		%\usepackage[onehalfspacing]{setspace}  	% eigentliche 1.5 Zeilenabstand

		\setlength{\parindent}{1.27cm}
		\usepackage{indentfirst} %um nach Überschrift 1. Indent nicht zu unterdrücken
		\renewcommand{\chapterheadstartvskip}{\vspace*{1\topskip}}	% Abstand von Kopfzeile zu Chapter-Überschrift
		\renewcommand{\chapterheadendvskip}{\vspace*{1\topskip}}	% Abstand von Chapter-Überschrift zu Text


% Überschriftenformatierung
		\renewcommand*{\chaptermarkformat}{\normalfont{ }}	% keine Nummerierung in Kopfzeile
		\renewcommand*{\raggedchapter}{\centering}			% Kapitelüberschriften mittig
		\RedeclareSectionCommand[beforeskip=-.25\baselineskip, afterskip=.25\baselineskip]{section}		% Abstand von Section Überschrift zu Text	
		\setkomafont{pagehead}{\normalfont{ }}		% Kopfzeilen vor Einleitung nicht kursiv
		
		%%%%%%SUSBSECTION 3. EBENE %%%%%%%%%
		%%%%%% VERSION Inline mit Einrückung %%%%%%%
			%\RedeclareSectionCommand[beforeskip=-.75\baselineskip, indent= 1.27cm, afterskip=-.5em]{subsection}		%afterskip: minus=In-Line Subsection Überschrift, Zahl = Abstand zum Folgetext,
		
		%%%%%% VERSION Outline ohne Einrückung %%%%%%%
			\RedeclareSectionCommand[beforeskip=-.25\baselineskip, afterskip=.25\baselineskip]{subsection}		%afterskip: minus=In-Line Subsection Überschrift, Zahl = Abstand zum Folgetext, 
		
		
% Schriftform & -größe
		\setkomafont{chapter}{\LARGE\rmfamily} 			% Überschrift der Ebene
		\setkomafont{section}{\Large\rmfamily}
		\setkomafont{subsection}{\large\rmfamily}
		\setkomafont{subsubsection}{\large\rmfamily}
		\setkomafont{chapterentry}{\large\rmfamily} 	% Überschrift in Inhaltsverzeichnis
		\setkomafont{descriptionlabel}{\bfseries\rmfamily} % für description-Umgebungen
		\setkomafont{captionlabel}{\small\bfseries}
		\setkomafont{caption}{\small}
		\usepackage{courier}							% für Zitation von Packages
		%\addtokomafont{disposition}{\Large} 			% Überschrift Abstract größer

% Tabellen
		\usepackage{multirow} 		% Tabellen-Zellen über mehrere Zeilen
		\usepackage{multicol} 		% mehrere Spalten auf eine Seite
		\usepackage{tabularx} 		% für Tabellen mit vorgegeben Größen
		\newcolumntype{C}{>{\centering\arraybackslash}X}	%centering Option in tabularx-float
		\usepackage{longtable} 		% Tabellen über mehrere Seiten
		\usepackage{array}			% zentrierte Elemente in Tabelle
		%\usepackage{rotating}     	% vertikaler Text in Tabellen
		\usepackage{booktabs}		% für Gestaltung horizontaler Linien
		\usepackage{chngcntr}		% counter für Tabellen-/Equation-/Figure-Nummerierung
			\counterwithout{table}{chapter}	% Tabellennummerierung unabhängig von Chapternr.
		\usepackage{caption}
		\DeclareCaptionLabelFormat{Nummerierung}{#1 #2}
		\captionsetup[table]{ 		% caption-package setup für Tab
			format=plain,  			% indention= generiert Einzug
			textfont={it, onehalfspacing, normalsize}, 	% Überschrift kursiv
			labelfont={normalfont,normalsize}, 			% Tabellennummerierung
			labelformat = Nummerierung,
			font=onehalfspacing,
			%width=.9\textwidth,	% feste Breite
			justification=justified,% Blocksatz
			singlelinecheck=false,	% damit Überschrift nicht zentriert, wenn einzeilig
			labelsep=newline,		% Abstand zw. Tabellennr. und Titel
			skip=0pt}				% Abstand zwischen Tabelle und Text davor
		\usepackage{threeparttable}
		% mglw. interessant: Use the siunitx package to align numbers with decimals
		\usepackage[table,xcdraw]{xcolor} 				% Farben

% Farben
		\definecolor{upblau}{HTML}{00305E} 				% uni potsdam blau
		\definecolor{upgrau}{HTML}{EBEBEB} 				% uni potsdam grau
		\definecolor{uporange}{HTML}{DF9945}			% humanwissenschaftsorange
		\definecolor{hrvbfborange}{HTML}{E46B0C}		% hrvbfb
		\definecolor{hrvbfbblau}{HTML}{00AFF4}  		% hrvbfb 
	

% PDF
		\usepackage[ngerman,pdfauthor={Neele Elbersgerd},  pdfauthor={Neele Elbersgerd}, pdftitle={Bachelorarbeit Psychologie}, pdfproducer	={LaTeX with hyperref},  
		breaklinks=true, 				% Zeilenumbruch bei links möglich
		bookmarks=true, 				% zeigt Lesezeichen in pdf an
		%colorlinks=true, linkcolor = up,	% wenn die Links farbig sein sollen
		linkbordercolor=upblau,				% für Farben der Kästen um Referenzen
		citebordercolor=upgrau,				% für Farben der Kästen um Zitationen
		urlbordercolor=uporange,			% für Farben der Kästen um URL-Links
		]{hyperref}
		\usepackage[final]{microtype} 	% mikrotypographische Optimierungen
		\usepackage{url} 				% ermöglicht Links (URLs)
		\usepackage{pdflscape} 			% einzelne Seiten drehen können
		\usepackage{pdfpages}      		% einbinden von pdf-Dateien




% Abbildungen
		\usepackage{graphicx} 			% Bilder
		\graphicspath{{images/}} 		% Standardpfad für Bilder
		\DeclareGraphicsExtensions{.pdf,.png,.jpg}	% bevorzuge pdf-Dateien vor den anderen
		\usepackage{subcaption}  					% mehrere Abbildungen übereinander
		\usepackage[all]{hypcap} 		% Klick auf Referenz führt zu Bild (nicht Caption)
		\usepackage{tikz}				% Zeichnungen in LaTeX
			\usetikzlibrary{positioning}
		\counterwithout{figure}{chapter}	% Abbildungsnummerierung unabhängig von Chapternr.
		\captionsetup[figure]{			% Captions bei Bildern
			format=plain,
			indention=0pt,				% kein Einzug
			labelsep=period,			% Punkt nach "Abbildung"
			labelfont=it,				% "Abbildung" kursiv
			margin={0pt,0.1\textwidth}, 	%rechter Rand, sodass Caption nur 90% von Textbreite
			justification=justified,
			singlelinecheck=false,
			font=onehalfspacing
		}
		%\setcapwidth[l]{0.9\textwidth} % Breite der Caption nur 90% der Textbreite
		%\setcapindent*{0em} 			% kein Einrücken der Caption von Figures und Tabellen
		\setlength{\abovecaptionskip}{0.3cm} % Abstand der zwischen Bild- und Bildunterschrift


%  Verzeichnisse & Bibliographie
		\setcounter{tocdepth}{\sectiontocdepth} 		% Gliederungstiefe ToC bis inkl. Section
		\usepackage{bibgerm} 							% Umlaute in BibTeX
		\usepackage[autostyle, german=quotes]{csquotes}	% deutsche Anführungszeichen
			
		%Anhang
		\newcommand*{\appendixmore}{		% siehe KOMA Doku S.339
			\renewcommand*{\chapterformat}{
				\appendixname~\thechapter\autodot\enskip}
			\renewcommand*{\chaptermarkformat}{
				\appendixname~\thechapter\autodot\enskip}}


%funktionierender BibLatex mit APA:	
		\usepackage[style=apa, 
		%apabackref=true, 			% Seitenangaben der Zitation im Verzeichnis
		backend=biber, 				% Biber als Backend
		hyperref=true, 				% Links in pdf
		%maxcitenames=6, maxnames=6, minbibnames=6, maxbibnames=6, % nur für bib
		apamaxprtauth=6, 			% max. 6 Namen, danach Auslassungszeichen
		%sorting = debug, 
		natbib=true, 
		language=ngerman, 
		doi=true, url=true, 
		%eprint=false,
		%useprefix=true,			
		uniquename=init 
		]{biblatex}
		\DeclareLanguageMapping{ngerman}{ngerman-apa}		% deutsceh Sprache
		\DefineBibliographyStrings{ngerman}{				% et al. darstellen
			andothers={et\ al\adddot}}	
		\addbibresource{./literature.bib} 			% Speicherort der Literatur
				
		%\DefineBibliographyExtras{ngerman}{\DeclareQuotePunctuation{.,}}
				

% Abkürzungsverzeichnis
		\usepackage{acronym}

% Quellcode		% für Formatierung in Quelltexten, hier im Anhang
		\usepackage{listings}
		\definecolor{grau}{gray}{0.25}
		\lstset{
			extendedchars=true,
			basicstyle=\tiny\ttfamily,
			%basicstyle=\footnotesize\ttfamily,
			tabsize=2,
			keywordstyle=\textbf,
			commentstyle=\color{grau},
			stringstyle=\textit,
			numbers=left,
			numberstyle=\tiny,
			breakautoindent  = true,		% für schönen Zeilenumbruch
			breakindent      = 2em,
			breaklines       = true,
			postbreak        = ,
			prebreak         = \raisebox{-.8ex}[0ex][0ex]{\Righttorque},
		}
		\usepackage{scrhack} 				% Vermeidung einer Warnung, die durch das Paket listings zum Anzeigen von Quelltext auftritt

% Fußzeile
		%\deffootnote{1.5em}{1em}{\makebox[1.5em][l]{\thefootnotemark}}	% linksbündige Fußnoten

% Eigene Befehle
	\newcommand{\leadingzero}[1]{							% führende Nullen (Datum)
		\ifnum #1<10 0\the#1\else\the#1\fi} 
	\newcommand{\todayI}{									% Datumsformat
		\leadingzero{\day}.\leadingzero{\month}.\the\year}			
	\newcommand{\mat}[1]{\mathbf{#1}} 						% Matrizen-Format
	\newcommand*{\ditto}{"}									% ditto marks
	\setcounter{equation}{0}					% für Referenzen der Gleichungen ab (1)
		\counterwithout{equation}{chapter}
		\renewcommand{\theequation}{\arabic{equation}}
		\newcommand{\nr}{\addtocounter{equation}{1}\tag{\theequation}}
	

%\typearea{14} % typearea berechnet einen sinnvollen Satzspiegel (das heißt die Seitenränder usw.) siehe auch http://www.ctan.org/pkg/typearea. Diese Berechnung befindet sich am Schluss, damit die Einstellungen von oben berücksichtigt werden
%\setlength{\parindent}{0pt}					% kein Einrücken im Text

