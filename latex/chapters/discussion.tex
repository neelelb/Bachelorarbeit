

	
	Das Ziel dieser Arbeit war der Vergleich der Verläufe von Schreck- und Hautleitwertreaktionen in einer Furchtakquisition durch ein multivariates Mehrebenenmodell. 
	Für diesen Zweck wurde ein Modell aufgestellt, um die Wachstumskurven der Indikatoren zu beschreiben und das differentielle Furchtlernen im Akquisitionstraining zu untersuchen. Die wichtigsten Erkenntnisse lassen sich wie folgt zusammenfassen: Zunächst kann aus den Ergebnissen kein Furchtlerneffekt, sondern lediglich ein Instruktionseffekt geschlussfolgert werden. Die Modellergebnisse sind damit zwar theoretisch und empirisch erwartbar, stellen aber für die Furchtlerntheorie keine wesentlich neuen Erkenntnisse bereit. Trotz alledem bieten Modelle wie diese substanzielle Vorteile und Chancen, in zukünftiger Forschung umfassendere Einblicke in die Prozesse des Furchtlernens zu gewinnen.


%%%%%%%%%%%%%%%%%%%%%%%%%%%%%%%%%%%%%%%%%%%%%
%%%%%%%%%%%% INTERPRETATION %%%%%%%%%%%%%%%%%
%%%%%%%%%%%%%%%%%%%%%%%%%%%%%%%%%%%%%%%%%%%%%

\section{Interpretation und Einordnung}	\label{interpretation}

	Im Fokus der Ergebnisse steht das aufgestellte Modell, aus dem zwei grundlegende Informationen über die Unterschiedlichkeit der Reaktionsvariablen hervorgehen. 
	Zunächst wurden die mittleren Verläufe beider Indikatoren über das Training hinweg modelliert. Während die SCR durch einen linearen Verlauf am besten beschrieben wurden, erwies sich für die Schreckreaktionen eine quadratische Modellierung als adäquater. Bei der Betrachtung der Rohverläufe wird deutlich, dass beide Reaktionsvariablen Habituationsprozessen unterliegen. Das heißt, dass die absoluten (bzw. log-transformierten) mittleren Reaktionen über die Zeit hinweg abnehmen. 
	Zusätzlich zur linearen Abnahme wurde bei der Schreckreaktion eine Linkskrümmung modelliert. Diese mag zwar signifikant zum Modell beitragen, hat aber im praktischen Sinne wenig Bedeutung für den Verlauf der Habituation, da der Grad der Krümmung gering ist gemessen an der absoluten Größe der Reaktionen. 
	Habituationen der Hautleitwert- und Schreckreaktionen insgesamt sind eine übliche Beobachtung \parencite[z.\,B.][]{BRADLEY1993b} und lassen sich durch die verwendeten Modellparameter angemessen beschreiben. 
	%ggf. noch was, dass man diese Terme auch als zufällig modellieren könnte, um Einblick über interindividuelle Unterschiede in Habituationsprozesse zu bekommen.
	%Habituation von Startle und SCR: Bacigalupo \& Luck, 2017; Bradley, Lang, \& Cuthbert, 1993; Grillon \& Baas, 2003)

	Der zweite -- und bedeutsamere -- Unterschied zeigt sich im Modell durch die Interaktion der linearen Zeitkomponente Trial und dem Stimulustyp. 
	Auffällig ist hier, dass der CS+/CS-- Unterschied über die Zeit hinweg bei der SCR als signifikanter Einfluss ins Modell einging, bei der Schreckreaktion jedoch nicht. 
	Die Interaktion kommt bei der SCR dadurch zustande, dass die Reaktionen auf den CS-- im Mittel abnehmen, während die Reaktionen auf den CS+ als nahezu konstant über die Trials modelliert werden. Diese Diskrimination spiegelt sich auch in den zeitpunktweisen Mittelwertvergleichen wider. Hier zeichnet sich bei den SCR in der zweiten Hälfte des Akquisitionstrainings zu Teilen ein substanzieller Mittelwertunterschied ab (für Trial $10$ bis $14$). 
	Für die Schreckreaktion hat die Interaktion währenddessen nicht signifikant zur Anpassungsgüte beigetragen. Zwar ist im Mittel die Reaktion auf den CS+ größer, allerdings ist ein Diskriminationseffekt über die Zeit nicht deutlich nachweisbar. Auch wenn in der zweiten Trainingshälfte eine Unterschiedlichkeit der Reaktionen auf die Stimuli zu zwei Zeitpunkten gegeben ist, so verdeutlichen die Modellergebnisse, dass ein Differenzierungseffekt über die Zeit fern von Eindeutigkeit bzw. von Unsicherheit geprägt zu sein scheint. 
	Der Vergleich der standardisierten Differenzen zwischen den Reaktionsmaßen %(Tabelle \ref{tab:descriptive2}) 
	bestätigt, dass auf dem Maß der SCR im Durchschnitt stärker zwischen CS+ und CS-- diskriminiert wird als auf der Schreckreaktion. %Zusammenfassend ist ein differentieller Effekt auf dem Maß der SCR stärker ausgeprägt als auf der Schreckreaktion.
	
	Im Großen und Ganzen liefern diese Ergebnisse keine Nachweise für ein differentielles Furchtlernen. 
	Eine Differenzierung, die sich in unterschiedlicher Stärke bei den beiden Reaktionsvariablen zeigt, tritt ausschließlich erst \textit{nach} dem Zeigen der Instruktion (zwischen dem achten und neunten Trial des jeweiligen CS) auf. Das lässt vermuten, dass die beobachteten Differenzierungen nicht auf eine über die Zeit erworbene Assoziation zurückgeführt werden können, sondern stattdessen durch die Kontingenzinstruktion bedingt sind. Nach einer aufklärenden Instruktion spricht man bei einer Differenzierung zwischen CS+ und CS-- von einem Furchtausdruck \parencite{LONSDORF2017fc} und nicht mehr von Furchtlernen.
	Vergangene Studien haben ähnliche Instruktionseffekte bereits in Furchtkonditionierungsstudien gezeigt. In der bereits referenzierten Studie von \textcite{WEIDEMANN2016} beispielsweise wurde eine Gruppe vollständig über die CS-US-Kontingenz aufgeklärt und zeigte sofort differentielle Schreckreaktionen, während eine uninstruierte Gruppe keinerlei CR erwarb. Bei einer dritten Gruppe, die nur über das Vorhandensein, nicht aber die Art der Kontingenz instruiert wurde, konnte über den Verlauf des Trainings ein zunehmender Furchtlerneffekt beobachtet werden. 
	Die Ergebnisse dieser Arbeit sind in Einklang zu bringen mit der Idee, dass eindeutige Kontingenzinstruktionen die CR verstärken oder zum Teil sogar auslösen können. 

	Vergleicht man qualitativ den Instruktionseffekt auf den beiden Reaktionsvariablen, fällt auf, dass der Furchtausdruck bei der Hautleitwertreaktion sehr eindeutig gezeigt wird. Die Evidenz für eine Differenzierung bei der Schreckreaktion ist aber weniger eindeutig und schwächer ausgeprägt.
	Hier ist auffällig, dass trotz einer vollständigen Instruktion über die CS-US-Kontingenzen in der Schreckreaktion keine beständige Differenzierung zwischen CS+ und CS-- beobachtet wird. 
	Ein Blick auf die zufälligen Effekte der Schreckreaktion macht deutlich, dass ein höherer Intercept (Reaktion auf den CS-- in der Mitte des Trainings) mit einem kleineren Interaktionseffekt dieser Versuchsperson einhergeht. Die Tendenz, dass eine höhere Reaktivität in der Mitte des Trainings mit einem geringeren Differenzierungsgrad zusammenhängt, ist möglicherweise interessant für zukünftige Forschung.
		%Auch wenn der Intercept keinesfalls als neutrale Grundmessung verstanden werden kann.
		%Vergangene Forschung hat gezeigt, dass Schreckreaktionen in experimentellen Aufgaben positiv mit der generellen Reaktivität in einer neutralen Grundmessung korreliert (CITE BRADFORD KAYE et al. 2014)
	Der Effekt der Instruktion schlägt sich auch in den Korrelationen zwischen den abhängigen Variablen nieder. 
	Eine substanziell positive Korrelation zwischen den CS+/CS-- Differenzwerten der beiden Maße wird nur direkt nach der Instruktion beobachtet. 
	Eine größere Differenzierung in der SCR geht in diesem Messzeitpunkt also mit einer größeren Differenzierung in der Schreckreaktion einher. 
	Die Instruktion erzielt also -- einem Beschleuniger gleich -- unmittelbar eine (mehr oder weniger starke) Diskrimination zwischen CS+ und CS-- über die nachfolgenden Trials.
	Überraschend ist, dass bei der SCR im Trial nach der Instruktion zusätzlich zu dem erwarteten Anstieg beim CS+ auch die Reaktion auf den CS-- deutlich ansteigt. 
	

 	Eine mögliche Erklärung für die beobachteten Ergebnisse ist der Einfluss des Untersuchsungsdesigns. 
	Insgesamt scheinen die Stimuli vor der Instruktion wenig prädiktive Aussagekraft zu besitzen. 
	Die niedrige Verstärkungsrate von \SI{50}{\percent} und die dadurch geringe Anzahl an US könnten die Ursache dieser Unsicherheit sein. In der Literatur werden experimentelle Situationen als \textit{schwach} bezeichnet, wenn sie durch große Komplexität, Unsicherheit und Ambiguität gekennzeichnet sind \parencite{LISSEK2006}. Es wird vermutet, dass diese schwachen Situationen sensitiver gegenüber interindividuellen Unterschieden in den Reaktionen sind \parencite{LONSDORF2017fc}. Die hohe postinstruktionale Reaktion auf den CS-- bei der SCR und die Uneindeutigkeit in der Differenzierung bei der Schreckreaktion können Hinweise auf eine von Unsicherheit geprägte Situation für die Versuchspersonen sein. Außerdem zeigen die signifikanten zufälligen Effekte ebenfalls, dass Versuchspersonen sich in ihrem Differenzierungsgrad über die Zeit wesentlich unterscheiden.
		
	 
	Ein weiterer potentieller Einflussfaktor knüpft an die geringe Verstärkerrate an: Im Untersuchungsdesign wurden die Versuchspersonen auf einer von vier verschiedenen Trialreihenfolgen getestet. In zwei der vier Reihenfolgen wird der erste CS+ nicht verstärkt. Die erste Präsentation einer CS-US-Paarung kann somit bei einigen erst zum dritten oder vierten Trial erfolgen.	
	Diese späte Präsentation %der zu erlernenden CS-US-Kontingenz 
	birgt das Potential, das kognitive Lernen zu erschweren, und kann in unterschiedlichen Lernkurven für die verschiedenen Gruppen resultieren \parencite{LONSDORF2017fc}. Das Phänomen einer verspäteten oder abgeschwächten Assoziationsbildung in Folge einer unverstärkten Exposition mit dem CS+ ist unter dem Namen \textit{latente Inhibition} bekannt. Es ist empirisch gezeigt worden, dass ein konsequenzloses Zeigen des späteren CS+ das Auftreten von differentiellen SCR in einer Furchtakquistion verzögert \parencite[z.\,B.][]{SIDDLE1985}. In ihrem Review zur latenten Inhibition beim Menschen fassen \textcite{VAITL1997} zusammen, dass der Effekt der latenten Inhibition umso stärker ist, je mehr Präexpositionstrials gezeigt werden. Beim elektrodermalen Reaktionsmaß sei ein expliziter Inhibitionseffekt im differentiellen Design erst nach mehr als 15 Trials zu beobachten. Dieser Wert überschreitet zwar weit die Anzahl an präexponierten CS in der hier untersuchten Studie, allerdings kann ein Einfluss der Trialreihenfolge nicht explizit ausgeschlossen werden. Ebenso führt die Analyse der Gelegenheitsstichprobe dazu, dass eine Ausbalancierung möglicher Reihenfolgeeffekte als kritisch zu betrachten ist.  %1: 10, 2:10, 3:8, 4:12
	
	
	Im Endeffekt spiegeln die Ergebnisse dieser Analyse die Besonderheiten des experimentellen Paradigmas wider, liefern aber keine Erklärungen für zugrundeliegende Lernprozesse oder das Zusammenspiel der beiden Faktoren in furchtlerntheoretischer Hinsicht. 
	Im Sinne der Zwei-Prozess-Theorie, die von einer kognitiven und einer automatischen Ebene des Furchtlernens ausgeht \parencite[z.\,B.][]{HAMM1996, OEHMANN2001}, wäre die Instruktion als Einfluss auf die kognitive Ebene zu verstehen.
	Unter der Annahme, dass die SCR ein Indikator dieser Ebene ist, sollte die Aufklärung über die CS-US-Kontingenz  vor allem mit einem verstärkenden Effekt auf die differentielle SCR einhergehen. Mit dieser Annahme lassen sich die Ergebnisse dieser Arbeit grundlegend vereinen. Währenddessen besteht auf einer automatischen Ebene -- mit der Schreckreaktion als Indikator --  im Mittel auch nach der Instruktion Unsicherheit. 
	Die grundlegende Einschränkung jeglicher Schlussfolgerungen aus diesen Daten ist das vollkommene Fehlen eines differentiellen Furchtlernens. Auch wenn die Modellunterschiede zwischen SCR und Schreckreaktion und der Einfluss der Instruktion potentiell vereinbar mit einer Zwei-Prozess-Theorie sind, macht es der fehlende Lerneffekt unmöglich, Aussagen über Furchtlernprozesse zu tätigen.
	Die Daten können in diesem Hinblick keine Indizien für die eine oder die andere Theorie zum Furchtlernen bereitstellen.
%FRAGESTELLUNG:
	Dieser Punkt wirkt sich auch auf die Einordnung und Beantwortung der leitenden Fragestellungen dieser Arbeit aus.
	Für die erste Frage untersuchte ich, ob sich unterschiedliche Verläufe in Hautleitwert- und Schreckreaktion abzeichnen, wenn beide Maße im Modell gleich behandelt werden. Zusammenfassend sind beide Verläufe durch Habituationsprozesse gekennzeichnet, wobei der Unterschied der polynomialen Modellierung als vernachlässigbar interpretiert wird. Qualitativ ist ersichtlich, dass die Reaktionen auf den CS-- über das Training hinweg bei beiden Maßen abnehmen, während bei dem CS+ nur die Schreckreaktionen als kontinuierlich fallend modelliert werden. Unterschiedliche mittlere Verläufe zeichnen sich dadurch im Modell durchaus ab, liefern aber wie gesagt keinen theoretischen Beitrag.
	Hier knüpft die zweite Fragestellung an, die beinhaltet, ob sich der Einfluss des Stimulustyps bedeutsam zwischen den abhängigen Variablen unterscheidet. 
	In der Tat konnten diskrete Unterschiede in der CS+/CS-- Diskrimination beobachtet werden, die oben bereits  detailliert dargelegt wurden. 
	Allerdings wurde auch gezeigt, dass die beobachtete Diskrimination der Stimuli auf Reaktionsebene nicht als Indikator einer gelernten Assoziation, sondern als Furchtausdruck in Folge einer Kontingenzinstruktion zu verstehen ist \parencite{LONSDORF2017fc}. Damit gibt es zwar Unterschiede im Einfluss des Stimulustyps auf SCR und Schreckreaktionen, jedoch keine Möglichkeit, Rückschlüsse auf verschiedene Furchtlernprozesse zu schließen.
	
	Trotz der geringen theoretischen Aussagekraft liefert das Modell durch seinen explorativen Charakter spannende Ein- und Ausblicke.
	Der zufällige Effekt der Interaktion \textit{lin$\times$CS} verdeutlicht, dass bei den Versuchspersonen signifikante Unterschiede in der CS+/CS-- Differenzierung über die Zeit hinweg auftreten -- und das auf beiden Reaktionsvariablen. Versuchspersonen unterscheiden sich also darin, ob und wie stark sie Furchtlernen bzw. einen Furchtausdruck zeigen. 
	Es ist bekannt, dass Individuen in ihrer physiologischen Reaktivität voneinander abweichen und ihnen eigene physiologische Antwortmuster aufweisen \parencite[z.\,B.][]{LACEY1958, ENGEL1960}. 
	Herkömmliche Auswertungsmethoden wie die rmANOVA kommen mit der Verwendung von Gruppenmittelwerten  für diese Idiosynkrasie nicht auf \parencite[z.\,B.][]{KRISTJANSSON2007}, Mehrebenenmodelle jedoch können diese Variationen systematisch analysieren. %Eine Repräsentation dessen, dass Versuchspersonen überhaupt unterschiedliche Responsivität aufweisen, wird über die zufälligen Intercepts realisiert. 
	Der explorative Charakter dieser Arbeit zeigt, dass diese möglicherweise durch die schwache Lernsituation hervorgehobenen interindividuellen Unterschiede sich durch das Modell erfassen und beschreiben lassen.
	Im multivariaten Teil des Modells -- nämlich die Korrelationen zwischen den zufälligen Effekten verschiedener Reaktionsmaße -- ist zwar kein Zusammenhang statistisch signifikant. Im Allgemeinen lassen sich hingegen aber Tendenzen ableiten, dass z.\,B. eine höhere Responsivität der SCR in der Mitte des Trainings \textit{potentiell} mit einer höheren Responsivität der Schreckreaktion einhergeht (Korrelation der zufälligen Intercepts). Im Folgenden werden Limitationen des Designs und des Modells dargelegt sowie ein Ausblick für die Anwendung von MEM gegeben.



%%%%%%%%%%%%%%%%%%%%%%%%%%%%%%%%%%%%%%%%%%%%%
%%%%%%%%%%%% LIMITATION %%%%%%%%%%%%%%%%%%%%%
%%%%%%%%%%%%%%%%%%%%%%%%%%%%%%%%%%%%%%%%%%%%%

\section{Limitationen und Ausblick}		\label{limitation}

	Bis hierhin wurde deutlich, dass das Ausbleiben eines Furchtlerneffekts zu großen Teilen durch das Studiendesign bedingt ist. 
	Dass die Daten für diese Arbeit im Rahmen eines Großprojekts mit anderer Zielsetzung als die der hiesigen Fragestellung erhoben worden sind, ist damit eine wichtige Limitation. 
%DESIGNOPTIMIERUNG
	Um konkreter Aussagen über Unterschiede der beiden untersuchten Reaktionsmaße im Furchtlernen treffen zu können, schlage ich eine Untersuchung mit folgenden Designoptimierungen vor: 
	In zukünftigen Studien könnten Hautleitwert- und Schreckreaktionen in einem uninstruierten Akquisitionstraining mit höherer Verstärkerrate (z.\,B. \SI{75}{\percent}) erhoben werden. 
	Akquisitionsdaten dieser Art wären aufschlussreich, um der eigentlichen Fragestellung dieser Arbeit näher auf den Grund zu gehen.
	Bei unterschiedlichen Trialreihenfolgen könnte in künftiger Forschung auch der Faktor der Präexposition mit dem späteren CS+ als Modellvariable  aufgenommen werden, um mögliche Einflüsse auf die Furchtlernverläufe explizit zu untersuchen.
	Im Sinne der Untersuchung, ob distinkte Furchtlernprozesse auf Reaktionsebene unterscheidbar sind, könnte das Hinzufügen von expliziten Bewusstseinsmaßen eine Option darstellen. Subjektive Maße zum Kontingenzbewusstsein, die verbales Wissen über die CS-US-Kontingenz operationalisieren und erfassen, können im Mehrebenenmodellrahmen die Forschung erweitern. Allerdings ist nicht außer Acht zu lassen, dass die Hinzunahme von abhängigen Variablen, die die Versuchsperson zusätzlichen Reizen aussetzen (wie der Schreckreiz oder Befragungselemente), potentiell auch das Furchtlernen beeinflussen \parencite[am Beispiel des Schreckreizes:][]{SJOUWERMAN2016}.
	Hier könnten zum Beispiel zwei verschiedene Gruppen mit den gleichen Akquisitionsparametern erhoben werden, von denen jedoch bei einer nur die Schreck- und die Hautleitwertreaktion erfasst werden, bei der anderen aber  zusätzlich Befragungselemente zum Kontingenzbewusstsein im Training eingebettet sind. So ließe sich der Einfluss der \textit{Befragung} immerhin mit modellieren.   
	Generell gilt, dass in der Zukunft multivariate Wachstumsmodelle genutzt werden können, um verschiedene Furchtindikatoren in ein einziges Modell zu integrieren. Der Vergleich nicht nur von Hautleitwert- und Schreckreaktionen, sondern auch von Maßen wie der Herzrate und der Pupillendilatation kann potentiell Einblick in die Interdependenzen des peripherphysiologischen Furchtsystems verschaffen. Korrelationen zwischen den zufälligen Effekten bieten die Möglichkeit, geteilte Variation zwischen den abhängigen Variablen zu erfassen \parencite[][]{CURRAN2012,MACCALLUM1997}. 
	Eine Einschränkung auf Modellebene ist hierbei aber grundlegend, dass eine steigende Komplexität des Modells auch mit der Notwendigkeit einhergeht, eine größere Stichprobe zu erheben.   
		%Außerdem kann im Kontext der \textit{Individual Response Stereotypy} \parencite[][]{ENGEL1960} auch die Unterschiedlichkeit von Versuchspersonen explizit mit einbezogen werden. 
		

	
		%Limitation: Garden of forking paths >> viele freie entscheidungen bei MLM, die man treffen muss. Standardisierung und Best-Practice empfehlungen helfen (Lonsdorf für Studiendesign, Meteyard für MLM, Ney?)
	Beim Einsatz von MEM gibt es -- wie immer in der empirischen Wissenschaft -- eine Reihe von Entscheidungen, die bis zum finalen Ergebnisbericht getroffen werden müssen. Zudem gibt es erst seit kurzem erste Schritte in Richtung einer Best-Practice für die Anwendung von MEM in der psychologischen Forschung \parencite{METEYARD2020}.
	Ich bin mir darüber bewusst, dass die Entscheidungen, die im Zuge der Modellbildung dieser Arbeit getroffen wurden, nur einen Weg unter vielen möglichen kennzeichnen. Angesichts dessen können meine Entscheidungen einen Einfluss auf die gefundenen Ergebnisse haben. 
	Zunächst ist hier die Aufbereitung der Daten anzubringen.  
	Während sich bei der SCR für eine logarithmische Transformation entschieden wurde, sind die Schreckreaktionen als Rohdaten in die Analyse eingegangen. Die Richtlinien von \textcite{BLUMENTHAL2005} verdeutlichen, dass bis jetzt keine Best-Practice-Lösung für die Transformation und Standardisierung von Schreckreaktionen festgelegt wurde. \textcite{BRADFORD2015} verglichen in Hinblick darauf rohe, standardisierte und prozentuale Quantifizierungen von Schreckreaktionen mit der Schlussfolgerung, dass die ersteren beiden vergleichbare Ergebnisse hervorbringen. Für die SCR wird demgegenüber eine Transformation (log oder Quadratwurzel) empfohlen, um die Reaktionsdaten zu normalisieren \parencite{BOUCSEIN2012}. Diese Empfehlungen bewegen sich auf einer univariaten Analyseebene und es bleibt zunächst offen, ob für multivariate Auswertungen im Sinne der Vergleichbarkeit anders gehandelt werden müsste. 
	Die Frage der Vergleichbarkeit spielt auch in der Interpretation der Modellergebnisse eine entscheidende Rolle: Der tatsächliche Vergleich der beiden abhängigen Variablen in dieser Arbeit beruht hauptsächlich auf qualitativen Modellierungsunterschieden. Um Unterschiede in \textit{Regressionskoeffizienten} zwischen zwei abhängigen Variablen auf unterschiedlichen Skalen zu quantifizieren, könnten standardisierte Regressionskoeffizienten in Frage kommen. Vor- und Nachteile verschiedener Standardisierungsverfahren zu diskutieren überschreitet den Rahmen dieser Arbeit, allerdings birgt dieser Ausblick gegebenenfalls wichtige Implikationen für multivariate Verfahren im Furchtkonditionierungskontext und sollte in zukünftige Forschung eingebunden werden.  
	Im Kontext der Reaktionsaufbereitung muss auch hervorgehoben werden, dass einer der wesentlichen Vorteile der Schreckreaktion -- nämlich ihre Erfassung ohne die Präsentation experimentell relevanter Stimuli -- in dieser Auswertung ungenutzt blieb. Im Sinne der Gleichbehandlung wurden die Schreckreaktionen aus den ITI als zusätzliche Informationsquelle außer Acht gelassen, was potentiell als Einschränkung der Datenqualität wertbar ist.
	
	Im Kontext des Modellbaus ist die Auswahl der Kovarianzstruktur anzusprechen, die als unstrukturiert (die Standardeinstellung im \texttt{nlme}-Paket) gewählt wurde. Bei Wachstumskurven wird manchmal eine autoregressive Korrelationsstruktur angenommen \parencite{BAGIELLA2000}. Um womöglich auch bei kleineren Datensätzen Konvergenz zu erzielen, wären Annahmen über eine passendere Kovarianzstruktur eine relevante Erweiterung.
	Eine wichtige Limitation des gebauten Modells ist die Verletzung der Normalverteilungsannahme der Residuen. %Zwar scheint das Modell die Daten logisch und angemessen zu erklären. 
	\textcite{MAAS2004} konnten zeigen, dass diese Verletzung auf der höchsten Gruppierungsebene wenig bis gar keinen Effekt auf die Schätzung der festen Effekte, jedoch aber einen Einfluss auf die Schätzung der Standardfehler der zufälligen Effekte hat.
	An den Residuen ist hier auch ersichtlich, dass das Modell nicht gut mit Nullreaktionen umgehen kann. Es gibt einen klaren Bodeneffekt (v.\,a. bei der SCR, wo der Anteil an Nullreaktionen größer ist) und zum Teil sagt das Modell negative Reaktionen vorher. Vor allem bei den vorhergesagten Intercepts für die Hautleitwertreaktion wird dieser Bias deutlich. 
	Alternative Ansätze zur klassischen Tiefpunkt-bis-Hochpunkt Definition der SCR können hier einen wertvollen Beitrag leisten. In der traditionellen Reaktionsdefinition führen überlagerte SCR zu einer Unterschätzung der Reaktionsamplitude \parencite{BENEDEK2010}. 
	Mittels eines Entfaltungsansatzes schlagen \textcite{BENEDEK2010} die Aufteilung von elektrodermaler Aktivität in phasische und tonische Komponenten vor (CDA) und zeigen, dass die dadurch verbesserte Tiefpunkt-bis-Hochpunkt Definition zu weniger Nullreaktionen und insgesamt höheren Amplituden der SCR führt. In zukünftigen Studien könnten Bodeneffekte mit diesem Wertungsverfahren minimiert werden.
	Analysetechnisch seien bei einer solchen Einschränkung ebenso Bootstrapping-Verfahren vorgeschlagen, um reliablere Schätzungen der Parameter zu erhalten \parencite{HOX2018}. Allerdings benötigen diese eine substanziell größere Stichprobe und gehen inhaltlich über den Rahmen dieser Arbeit hinaus. 
	
	Leider ist das Modell nur mit einer kleinen Teilmenge der möglichen zufälligen Effekte konvergiert. Eine Ursache dafür kann die bereits angesprochene -- gemessen an der Modellkomplexität kleinen -- Stichprobe sein \parencite{METEYARD2020}. Vollständigere Einblicke in interindividuelle Unterschiede der Versuchspersonen auf den Zeitkomponenten ließen sich vielleicht bei einer größeren Stichprobe (mind. $\geq 50$) gewinnen.
	Potentiell bietet sich für zukünftige Forschung an, in einem explorativen Ansatz bereits erhobene Daten mit MEM zu reanalysieren, um spätere konfirmatorische Studien anzustoßen. 
	
	
	
	
	Abschließend wird trotz bestehender Limitationen deutlich, dass Mehrebenenmodelle nicht zu unterschätzende Vorteile für die Auswertung solch komplexer Daten haben, wie sie in Furchtkonditionierungsstudien zu finden sind. 
	Zum Beispiel diskutieren \textcite{NEY2018} in ihrem Artikel verschiedene Probleme und Limitationen in der herkömmlichen statistischen Auswertung von Furchtextinktionsdaten. Ihre Schlussfolgerung legt dar, dass statistische Modellierung (z.\,B. durch MEM realisiert) eine wichtige Weiterentwicklung ist, um unter dem Schatten %xxx besseres Wort? Drohung? Mantel?
	der Replikationskrise die Reliabilität und Validität von Ergebnissen und Aussagen zu erhöhen \parencite{NEY2018}.
	In diesem Sinne bin ich der Meinung, dass auch wenn die in dieser Arbeit präsentierten Ergebnisse lerntheoretisch kaum neue Informationen bereitstellen, so doch die vorgestellte Methodik vielleicht Forscherinnen und Forscher zu motivieren vermag, im Kontext der Furchtkonditionierung konkreten multivariaten Fragestellungen nachzugehen. 
	Durch die simultane Auswertung von Hautleitwert- und Schreckreaktionen können zukünftige konfirmatorische Studien einen Beitrag zum Diskurs um die Ein- und Zwei-Prozess-Theorien leisten. 
 	Letztendlich birgt der Einsatz integrativer Auswertungsverfahren das Potential und die Hoffnung, bestehende  Limitationen langfristig zu überwinden, unsere Theorien rund um das Furchtlernen weiterzuentwickeln und seine Mechanismen zu durchdringen. 



%Es ist meine Meinung, dass der Einsatz integrativer Auswertungsverfahren und die Überwindung bestehender Limitationen langfristig notwendig sind, um unsere Theorien rund um das Furchtlernen weiterzuentwickeln und seine Mechanismen zu durchdringen.


	%TRANSFORMATION 
%Blumenthal: 
%For reasons as yet largely unknown, wide individual differences in absolute blink magnitude are observed, and this variation is often unrelated to the experimental phenomena of interest. Accordingly, the use of absolute blink magnitudes can result in a small number of subjects with unusually large blinks disproportionately affecting the outcome. For this reason, many experimenters standardize blink magnitudes in some way, such as using all blinks for a given subject as the reference distribution and reporting the results as z or T (mean 5 50, SD 5 10) scores. An alternative is to use only blinks obtained during intertrial intervals, or other nontask parts of the session (control blinks), as the reference distribution (e.g., Bonnet, Bradley, Lang, & Requin, 1995)
%Although no preferred method for standardization or rejection of outliers has emerged, any such data transformation should be reported in detail.
%Bradford:
%Nonetheless, raw and standardized startle response approaches will produce comparable results when the size of within-subject effects is consistent despite individual differences in response variance.
%General startle reactivity. Startle response in experimental tasks is strongly positively related to general startle reactivity measured during a baseline procedure (Bradford, Kaye, & Curtin, 2014). recent theory and empirical evidence suggests that general startle reactivity may index individual differences in defensive reactivity to aversive stimuli generally (Bradford, Kaye, & Curtin, 2014; Vaidyanathan et al., 2009)
%Contradictory to correlations i found
%However, it should be noted that the magnitude of raw potentiation is typically greater for participants with higher startle response in neutral conditions (Bradford, Kaye, & Curtin, 2014).


%%%%%%%%%%%%%LEITFADEN BACHELORARBEIT

	%Zunächst erfolgt eine Zusammenfassung der wichtigsten Ergebnisse. In der Regel ist es dann günstig, die gefundenen Resultate pro Fragestellung unter einer entsprechenden Überschrift zusammenzufassen und zu interpretieren. Dabei sollen die in der Einleitung, der Theorie und der Fragestellung dargestellten Gedankengänge wieder aufgenommen und eine Kontinuität erreicht werden.
	%
	%Stehen die eigenen Ergebnisse im Widerspruch zu früheren Publikationen ist zunächst zu prüfen, ob sich die Abweichungen durch Unterschiede in der Operationalisierung der Variablen erklären lassen. Bei Unterschieden in der Operationalisierung ist zu begründen, welche eher dazu geeignet ist, das betreffende theoretische Konstrukt messbar zu machen. Bei Übereinstimmung in der Operationalisierung erfolgt ein Vergleich der methodischen Qualität der eigenen mit früheren Arbeiten um zu entscheiden, bei welchen Ergebnissen eine Replikation wahrscheinlicher ist.
	%
	%Dann wird geprüft, ob die Ergebnisse im Einklang oder im Widerspruch zu theoretischen Vorhersagen stehen. Falls die Ergebnisse im Widerspruch zu theoretischen Vorhersagen stehen, wird geprüft, ob die Ergebnisse eher im Einklang mit existierenden alternativen Theorien stehen. Zudem kann geprüft werden, ob die gefundenen Effektstärken es erlauben, eine plausible Modifikation der Theorie vorzuschlagen. Wenn das der Fall ist, erfolgt ein Ausblick, wie diese Modifikation durch zukünftige Untersuchungen geprüft werden kann.
	%
	%Limitationen der eigenen Arbeit, d.h. Grenzen der Generalisierbarkeit und Interpretierbarkeit, sind unter einer separaten Überschrift darzustellen. Schließlich werden die wichtigsten Schlussfolgerungen der Arbeit noch einmal zusammengefasst und deren Bedeutung für Praxis, Theorieentwicklung und weitere empirische Arbeiten herausgestellt.\\
	%
	%\textbf{Leitfragen:} Liegt eine erkennbare Trennung von Ergebnissen und Interpretation vor? Werden die Ergebnisse integriert, d.h. werden Einzelergebnisse aufeinander, aber auch auf frühere empirische Befunde und theoretische Annahmen bezogen? Wird der eigene Untersuchungsansatz kritisch reflektiert? Werden die eigenen Ergebnisse angemessen generalisiert? Werden Ansätze zu Folgeuntersuchungen diskutiert?