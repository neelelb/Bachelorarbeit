%%%%%%%%%%%%%%%%%%%%%%%%%%%%%%%%%%%%%%%%%%%%%
%%%%%%%%%%%EINLEITUNG%%%%%%%%%%%%%%%%%%%%%%%%%%%%
%%%%%%%%%%%%%%%%%%%%%%%%%%%%%%%%%%%%%%%%%%%%%

	Wenn wir Furcht empfinden, dann zumeist als Reaktion auf einen potenziell gefährlichen Reiz in unserer direkten Umwelt. Unser Körper begibt sich in einen erregten, defensiven Zustand und erhöht die Verhaltensbereitschaft für Flucht- oder Abwehrreaktionen. Furcht ist damit eine sehr adaptive und evolutionär relevante Emotion, die den Organismus vor Gefahr bewahren und möglichst schnell eine schützende Reaktion initiieren soll. In der psychologischen Forschung wird die Emotion Furcht typischerweise durch eine hohe Erregung und eine negative Valenz charakterisiert \parencite{LANG1995}.
	Welche Reize in der Umwelt als bedrohlich wahrgenommen werden, ist von verschiedenen Faktoren abhängig. Mittlerweile ist aber klar, dass Lernprozesse eine substanzielle Rolle dabei spielen. Unsere vergangenen Erfahrungen lenken unsere Wahrnehmung, Verarbeitung und Reaktion heute. Dieses erfahrungsbasierte Lernen -- oder \textit{Konditionierung} -- ist eine der Grundlagen für die Plastizität des Furchtsystems: Reize, die zuvor mit furchterregenden Ereignissen in Verbindung gebracht wurden, sind später in der Lage, selbst Furcht hervorzurufen. 

	In der psychologischen Grundlagen- und Anwendungsforschung werden Experimente zur Furchtkonditionierung seit knapp einem Jahrhundert für viele Fragestellungen eingesetzt. Die  Erkenntnisse führten in verschiedenen Disziplinen der Humanwissenschaften zu einem Boom an Abwandlungen des Paradigmas und Weiterentwicklungen in multiplen Forschungszweigen \parencite{MERZ2020}. Konzeptuell lassen sich Furchtkonditionierungsstudien in experimentelle Phasen unterteilen, die verschiedene Teilprozesse wie Furchterwerb, Extinktion und Wiederauftreten von Furcht zu modellieren versuchen. Innerhalb der Forschung um die Entstehung und Aufrechterhaltung von Angst- und Belastungsstörungen haben Furchtkonditionierungsexperimente entscheidende Einblicke in die Lern- und Gedächtnisprozesse von Menschen liefern können \parencite[für eine Übersichtsarbeit siehe][]{MINEKA2008}. Ebenso relevant sind sie für die (Weiter-)Entwicklung von Therapieelementen bei diesen Störungsbildern. Gerade dieser translationale Wert von Furchtkonditionierungsstudien trägt wesentlich zu ihrer unbestrittenen Relevanz bei.

	Die Fülle an Anwendungsgebieten mündete in einer Vielzahl an methodischen und terminologischen Variationen. Um eine bessere Vergleichbarkeit zu gewährleisten und Konsistenz in der Forschung zu fördern, haben \textcite{LONSDORF2017fc} Empfehlungen zur Methodik und Terminologie in Furchtkonditionierungsstudien zusammengestellt. Es sei zu Beginn darauf hingewiesen, dass methodische und technische Termini dieser Arbeit -- soweit nicht anders definiert -- im Einklang mit der Handreichung von \textcite{LONSDORF2017fc} zu verstehen sind. 

%%%%%%%%%%%%%%%%%%%%%%%%%%%%%%%%%%%%%%%%%%%%%
%%%%%%%%%%%DIE FURCHTKONDITIONIERUNG%%%%%%%%%%%%%%%%%
%%%%%%%%%%%%%%%%%%%%%%%%%%%%%%%%%%%%%%%%%%%%%

\section{Das Paradigma der Furchtkonditionierung}	\label{fearcondi}

	\subsection{Grundlagen der Klassischen Konditionierung}	\label{classic}

		Die Furchtkonditionierung beruht grundlegend auf dem Paradigma der Klassischen Konditionierung (auch Pavlovsche Konditionierung genannt). In seiner Pionierarbeit berichtete der russische Forscher \textcite{PAVLOV1927} von einer Konditionierung an einem Hund, an dessen Beispiel das Grundprinzip nachfolgend erläutert wird. 
		Die Klassische Konditionierung nutzt eine bereits bestehende Verbindung zwischen einem sogenannten Unkonditionierten Stimulus (US; hier Futter) und einer Unkonditionierten Reaktion (UR; hier erhöhter Speichelfluss beim Hund). Ein zunächst neutraler Stimulus -- bei Pavlov ein Glockenton -- löst wiederum eine neutrale Reaktion aus. In einem sogenannten \textit{Akquisitionstraining} wird der Glockenton wiederholt mit dem Futter (US) gemeinsam präsentiert, worauf der Hund mit erhöhtem Speichelfluss (UR) reagiert. Die Präsentation einiger dieser Paarungen führt dazu, dass der Glockenton zu einem \textit{Konditionierten Stimulus} (engl.: conditioned stimulus; CS) wird. Nach erfolgreichem Lernen ist die alleinige Präsentation des CS ausreichend, um den Speichelfluss auszulösen, der nun als \textit{Konditionierte Reaktion} (engl.: conditioned response; CR) bezeichnet wird. Diese CR ist in ihrer Modalität der UR meist ähnlich -- sprich, sie beschreiben das gleiche Verhalten -- allerdings ist die CR oft weniger intensiv.
		
		Die Furchtkonditionierung ist eine Variante dieses klassischen Paradigmas. Hier wird ein meist neutraler CS mit einem negativ valenten US (z.\,B. einem elektrotaktilen Stimulus) gepaart. Diese aversiven US lösen beim Individuum eine meist automatische (Furcht-)Reaktion wie den Schreckreflex aus. Diese Arbeit betrachtet die Akquisitionsphase einer Furchtkonditionierungsstudie. Hier wird mittels eines Akquisitionstrainings (der wiederholten gemeinsamen Präsentation von CS und US) eine konditionierte Reaktion manifestiert. Konzeptuell induziert dieses Training den Prozess des \textit{Furchtlernens}. 


	\subsection{Kognitive Erweiterungen}	\label{cognitive}

		Der Konditionierungsprozess wurde traditionell als simple und automatische Assoziationsbildung angesehen, die reflexiv und unfreiwillig erworben und abgerufen wird. Erst in den 70er Jahren wurde mit der \textit{kognitiven Wende} in der psychologischen Forschung den kognitiven Prozessen ein größerer Stellenwert eingeräumt. 
		Zum Beispiel zeigte \textcite{RESCORLA1968} in seiner Arbeit, dass die Kontingenz zwischen CS und US eine ausschlaggebende Rolle für den Erwerb einer CR spielt. Zudem lieferte er einen ersten Beleg dafür, dass die CS-US-Assoziation am besten gelernt wird, wenn der CS kurz vor dem US präsentiert wird. Diese Form der zeitlichen Kontiguität wird in der Furchtkonditionierungsforschung als \textit{Verzögerungskonditionierung} bezeichnet.
		Kognitive Ansätze dieser Art von Rescorla und anderen führten zu der weithin vertretenen Annahme, dass sich nach erfolgreichem Furchtlernen im Individuum die Erwartung an den US bildet, sobald ein CS dargeboten wurde. Diese US-Erwartung rückt die CR in die Position einer provisorischen Schutzreaktion auf Basis der gelernten Erfahrung, dass der US bald folgen wird \parencite{KIESEL2012}. Anders gesagt, kommt dem CS eine Prädiktorfunktion für den US zu. 
			
		Neben dem assoziativen Lernen im Sinne direkt erlebter Erfahrungen werden auch andere Pfade für das Furchtlernen in Betracht gezogen. \textcite {RACHMAN1977} postulierte neben der Konditionierung auch den Furchterwerb über Beobachtung und über Instruktionen. Beim beobachtenden Lernen observiert die Versuchsperson zum Beispiel die Furchtreaktion eines anderen Individuums auf einen Reiz, während sie beim instruktionalen Furchtlernen über die kausale Beziehung zwischen CS und US informiert wird. 
		Auch wenn dieser theoretische Ansatz nicht Gegenstand dieser Arbeit ist, gehe ich auf das instruktionale Furchtlernen kurz ein, da das verwendete Studiendesgin Instruktionselemente enthält (Abschnitt \ref{acq}).
		
		Instruktionen können die Versuchssubjekte teilweise oder vollständig über die vorhandenen CS-US-Kontingenzen aufklären. Während bei uninstruierten Designs erfahrungsbasiertes Furchtlernen über mehrere CS-US-Paarungen hinweg stattfinden kann, führt die explizite Aufklärung über die CS-US-Kontingenz sofort zu einer veränderten Furchtreaktion. Man spricht in diesem Fall von einem \textit{Furchtausdruck} statt von Furchtlernen \parencite{LONSDORF2017fc}. Diese erzwungene Bewusstwerdung kann einen erheblichen Einfluss auf den Lernprozess haben. Zum Beispiel zeigten \textcite{WEIDEMANN2016}, dass verbale Instruktionen über das Vorhandensein einer Beziehung zwischen CS und US das Auftreten von konditionierten Reaktionen (gemessen über die Lidschlusskomponente der Schreckreaktion) im Vergleich u.\,a. zu einer uninstruierten Gruppe erhöhen. Eine kürzlich erschienene Übersichtsarbeit von \textcite{MERTENS2018a} beleuchtet den moderierenden Effekt verbaler Instruktion auf die verschiedenen Phasen der Furchtkonditionierung tiefergehend (zur instruierten Extinktion sei auch die Übersichtsarbeit von \nptextcite{LUCK2016}, empfohlen). 
	
	
	
	\subsection{Das Differentielle Paradigma}	\label{differential}
	
		In der Furchtkonditionierungsforschung ist das sogenannte \textit{differentielle} Design eine relevante Erweiterung. Hierbei gibt es einen CS+, der immer oder manchmal mit einem US gepaart wird, und einen CS--, der niemals gemeinsam mit einem US auftritt. Typischerweise löst der CS+ eine stärkere CR aus und wird negativer bewertet als der CS--. Die Quantifizierung der Reaktionen kann zum Beispiel als differentielle CR erfolgen, sprich als Differenz der Reaktionen auf CS+ und CS--. In den überwiegenden Fällen sind CS+ und CS-- Reize ähnlicher oder gleicher Modalität (z.\,B. zwei unterschiedliche visuelle Stimuli). Die paradigmatische Ergänzung bietet mehrere Vorteile. Sie gewährleistet hinsichtlich des Erwerbs der CS-US-Kontingenz, dass der prädiktive Wert eines bestimmten Reizes gelernt und nicht das Vorhandensein eines Reizes an sich als Prädiktor missverstanden wird. Außerdem erlaubt das differentielle Konditionieren eine Kontrolle von möglichen Sensibilisierungseffekten. Das heißt, dass Effekte aufgrund der Modalität der CS vergleichbar zwischen CS+ und CS-- sind. 

	
%%%%%%%%%%%%%%%%%%%%%%%%%%%%%%%%%%%%%%%%%%%%%
%%%%%%%%%%%INDIKATOREN DES FURCHTLERNENS%%%%%%%%%%%%%%
%%%%%%%%%%%%%%%%%%%%%%%%%%%%%%%%%%%%%%%%%%%%%
	
\section{Indikatoren des Furchtlernens}	\label{index}
	
	Emotionen und Lernen haben als psychologische Konzepte eine wesentliche Sache gemeinsam: Auch
	wenn man eine Vorstellung von ihnen hat und sie relativ gut erkennt, so ist es gar nicht so leicht, sie richtig zu messen. Um das zu tun, müssen sie operationalisiert werden. 
	%Es fällt zum Teil leicht, sie zu erkennen, aber sie müssen operationalisiert werden, um quantifizierbar zu sein. 
	Beide Konstrukte sind komplexe Phänomene, die mit vielfältige Reaktionen auf verschiedenen Reaktionsdimensionen auslösen können.
	Diese können erfasst werden und ermöglichen es, Rückschlüsse auf eine bestimmte Emotion oder einen erfolgten Lernprozess zu ziehen.
	Furcht ist eine Emotion, die mit einer Aktivierung des Defensivnetzwerks einhergeht. Wenn Menschen Furcht empfinden, gibt es -- abhängig von der Proximität des furchtauslösenden Reizes -- eine Reihe an körperlichen Systemen, die in Gang gesetzt werden \parencite[zum menschlichen Defensivsystem siehe][]{HAMM2005, HAMM2006, OEHMANN2001}. Außerdem sind Menschen in der Lage, mitzuteilen, wie sich Furcht anfühlt. Ganz ähnlich ist das beim Lernen. In \textcite{KIESEL2012} wird Lernen definiert als "`Prozess, der als Ergebnis von Erfahrungen relativ langfristige Änderungen im Verhaltenspotential erzeugt"' (S. $11$). Allerdings reicht eine Änderung im Verhaltens\textit{potential} nicht, um Lernen experimentell nachweisen zu können. Es muss eine tatsächliche Änderung im Verhalten stattfinden, damit Rückschlüsse auf einen zugrundeliegenden Lernprozess gezogen werden können.
	
	Um nun Furchtlernen im Labor zu erfassen, kommen grundlegend drei Reaktionssysteme in Frage, die von \textcite{BRADLEY2000} in der Humanforschung unterschieden werden. Das sind (a) Handlung oder die Ausführung behavioraler Sequenzen (z.\,B. Flucht), (b) emotionale Sprache und (c) physiologische Reaktionen (z.\,B. steigende Herzrate). Die Veränderungen in Verhalten, physiologischen Reaktionen oder verbalen Berichten im Verlauf der Zeit sind damit klassische abhängige Variablen in Furchtkonditionierungsexperimenten.
	Bezogen auf die Furchtkonditionierung werden behaviorale Maße eher selten verwendet, am ehesten in Reaktionszeitexperimenten oder mittels Beobachtung von Vermeidungsverhalten. Verbale Maße sind häufiger: Sie können erfassen, inwieweit Subjekte explizites Wissen über CS-US-Kontingenzen erwerben ("`Der Reiz A sagt den Reiz B vorher"') oder wie die Bewertung von Affekt und Erregung sich im Laufe eines Experimentes ändert. 
	Physiologische Maße haben aufgrund ihrer Vergleichbarkeit mit Ergebnissen aus Tierstudien eine besonders wichtige Stellung in der Konditionierungsforschung. Unter den am häufigsten verwendeten Korrelaten sind peripherphysiologische Indikatoren wie die Herzrate, die Hautleitfähigkeit und die Schreckreaktion sowie neurobiologische Maße wie BOLD-Kontraste im fMRT oder Ereigniskorrelierte Potenziale im EEG. Die physiologischen Maße, vor allem diejenigen, die die Aktivität des autonomen Nervensystems (ANS) erfassen, gehen mit Vor- und Nachteilen einher. Ihr größtes Potential liegt darin, dass sie parallel zum Lernprozess erhoben werden können. Damit liefern sie einzigartige Informationen über körperliche Zustände, die zeitlich kontingent zum tatsächlichen Lernen geschehen. Allerdings sind sie -- anders als beispielsweise Ratings -- nicht oder zumindest nicht gleichermaßen furchtselektiv, was Inferenzen auf einzelne psychologische Prozesse erschwert (Abschnitte \ref{scrinference} und \ref{startleinference}).
	Zwei der am häufigsten verwendeten physiologischen CR sind die Lidschlusskomponente der Schreckreaktion und die Hautleitwertreaktion. Die folgenden Abschnitte erklären, wie sie erfasst werden, welche Körperprozesse sie abbilden und warum ihnen eine besondere Bedeutung zukommt.
	

%%%%%%%%%%%%%%%%%%%%%%%%%%%%%%%%%%%%%%%%%%%%%
%%%%%%%%%%%%%%%HAUTLEITWERTREAKTION%%%%%%%%%%%%%%%%
%%%%%%%%%%%%%%%%%%%%%%%%%%%%%%%%%%%%%%%%%%%%%

\section{Die Hautleitwertreaktion}		\label{scr}

	\subsection{Aspekte der Methodik}	\label{scrmethods}

		Die Hautleitwertreaktion (engl.: skin conductance response; SCR) ist ein Maß der Elektrodermalen Aktivität (EDA) und quantifiziert eine kurzfristige Änderung in der Leitfähigkeit der Haut in Reaktion auf einen Stimulus oder ein Ereignis. Sie ist ein weit verbreitetes und etabliertes Maß der Psychophysiologie. 
		Die Leitfähigkeit der Haut wird über Schweißdrüsen vermittelt und kann dort leicht und non-invasiv erfasst werden. Ekkrine Schweißdrüsen befinden sich fast überall am Körper, wobei ihre Dichte in der Handinnenfläche und an den Fingern mit am größten ist. Ableitungen der EDA werden präferiert dort vorgenommen \parencite{BOUCSEIN2012, DAWSON2007b}. In methodischer Hinsicht liegen die Vorteile der EDA darin, dass sie risikofrei, günstig und einfach zu erheben ist im Vergleich zu aufwendigeren physiologischen Methoden wie zum Beispiel dem EEG. Allerdings sind Hautleitwertreaktionen mit ihren $1$ bis \SI{4}{\second} Latenzzeit ein relativ langsames Maß \parencite{BOUCSEIN2012, DAWSON2007b}. Abgesehen von der Reaktionszeit unterscheidet sich die SCR von der Schreckreaktion auch dahingehend, dass kein zusätzlicher Auslösereiz benötigt wird, um sie zu erfassen (siehe Abschnitt \ref{startle} zur Schreckreaktion). 
		Im differentiellen Furchtkonditionierungsparadigma wird die SCR als Differenzmaß analysiert. Hier zeigte sich in zahlreichen Studien, dass durch erfolgreiches Lernen die Präsentation eines CS+ typischerweise eine höhere phasische Reaktion des Hautleitwerts hervorruft als die Präsentation eines CS-- \parencite{DAWSON1973, SEVENSTER2014}.
		Aufgrund der Latenzzeit der SCR muss bei dem Design von Furchtkonditionierungsstudien darauf geachtet werden, dass die Intertrial-Intervalle (ITI) zwischen zwei präsentierten Stimuli lang genug sind, damit die Hautleitwertreaktion auftreten und wieder abklingen kann \parencite{LONSDORF2017fc}.
		
		
	\subsection{Verwendung als Indikator}		\label{scrinference}
		%alternativ: „mögliche Inferenzen“
		
		Elektrodermale Maße werden in psychologischen Experimenten zur Erfassung vieler latenter Variablen und Prozesse herangezogen. Hautleitwertreaktionen sind als reliable Komponente der Orientierungsreaktion in Gegenwart von neuen, unerwarteten und signifikanten Stimuli \parencite{SIDDLE1991} weit verbreitet. Typischerweise wird beobachtet, dass die SCR bei wiederholter Präsentation in ihrer Amplitudenhöhe abnimmt. Dieser Habituationsverlauf geschieht zum Teil langsamer, je signifikanter (wichtiger) der Stimulus für das Individuum ist \parencite{BOUCSEIN2012}. Ein Beispiel für diese Signifikanz ist etwa die Relevanz des CS+ als Prädiktor für den US in einer Konditionierungsstudie. Zudem spielen auch Aufmerksamkeitseffekte und autonome Erregung eine nicht zu vernachlässigende Rolle in der Modulation von Hautleitwertreaktionen \parencite[für eine Ausführung siehe][]{DAWSON2007b}.
		
		Die Veränderung der EDA tritt weder in Isolation auf noch ist sie spezifisch für eine bestimmte Art von Reiz oder Kontext. Das bedeutet, dass die SCR oft nur eine Komponente in einer komplexen Reaktionskette des ANS abbildet. Diese kann wiederum durch verschiedene Einflüsse ausgelöst und moduliert werden, von Kontextbedingungen über Stimuluseigenschaften bis hin zum motivationalen und emotionalen Zustand, in dem sich die Versuchsperson befindet. Es ist typisch, dass Reaktionen, die die EDA beeinflussen, auch in andere Maße des ANS wie die Herzrate intervenieren. Tatsächlich müssen aber nicht alle diese Maße miteinander hoch korrelieren. Das liegt daran, dass Individuen immer wieder ein ihnen eigenes, relatives Reaktionsmuster des ANS produzieren und sich diese Muster aber interindividuell unterscheiden. Ein Phänomen, das als \textit{Individual Response Stereotypy} bekannt ist \parencite{ENGEL1960} und einen Grund dafür darstellt, dass für die Aktivierung des autonomen Nervensystems in der Forschung mehrere Indikatoren herangezogen werden. 
		
		Für sympathische Aktivität gilt die EDA hingegen als sehr spezifisches Maß. Ekkrine Schweißdrüsen und damit die EDA werden -- anders als andere Maße des ANS wie die Herzrate -- fast ausschließlich durch den Sympathikus kontrolliert \parencite{DAWSON2007b}. Die EDA kann dadurch mit der sympathischen Aktivierung infolge einer Defensivreaktion in Verbindung gebracht werden \parencite{MERZ2020}. Besonders responsiv zeigt sich die EDA in Situationen mit diskreten furchtauslösenden Stimuli, in denen eine Vermeidung nicht möglich ist \parencite{DAWSON2007b}. 
		Verwendet man die EDA nun als Indikator für Furchtlernen, bleibt jedoch das Problem der geringen Spezifität bestehen. Zwar konnten Studien zeigen, dass die SCR linear mit steigender induzierter Erregung der Stimuli ansteigt \parencite{LANG1990}, allerdings spielt sich dieser Effekt unabhängig von der Valenz der Stimuli ab. Im Kontext einer Konditionierung bedeutet das, dass ein CS gepaart mit einem aversiven US (Furchtkonditionierung) dasselbe Reaktionsmuster der EDA auslöst wie ein CS, der mit einem nicht-aversiven US gepaart wird \parencite{HAMM1996}.
		
		Ihre fehlende Spezifität wird der EDA häufig als Nachteil ausgelegt, denn durch ihre multikausale Bedingtheit ist sie kaum spezifisch interpretierbar. Trotz alledem ist ihre Bedeutung nicht von der Hand zu weisen. In den meisten experimentellen Designs hat das Ausmaß der Kontrolle einen entscheidenden Einfluss darauf, wie überzeugend und konkret SCR interpretiert und Inferenzen auf bestimmte psychologische Prozesse gezogen werden können \parencite{DAWSON2007b}. 
		
		
%%%%%%%%%%%%%%%%%%%%%%%%%%%%%%%%%%%%%%%%%%%%%
%%%%%%%%%%%%%%%SCHRECKREAKTION%%%%%%%%%%%%%%%%%%%
%%%%%%%%%%%%%%%%%%%%%%%%%%%%%%%%%%%%%%%%%%%%%


\section{Die Lidschlusskomponente der Schreckreaktion}		\label{startle}
		
	\subsection{Aspekte der Methodik}				\label{startlemethods}
		
		Die Schreckreaktion (engl.: startle reflex) beschreibt eine Reihe von protektiven motorischen Reflexen, die in Reaktion auf einen abrupten sensorischen Stimulus (engl.: probe stimulus; im Folgenden: Schreckreiz) ausgelöst werden \parencite{LANDIS1939}. Dieses Reaktionsmuster ist speziesübergreifend feststellbar und seine neuronalen Schaltkreise wurden in Tierstudien bereits eingehend erforscht und zum Teil in der Humanforschung repliziert \parencite[für eine Übersichtsarbeit siehe][]{DAVIS2006}. In letzterer erweist sich die Lidschlusskomponente (engl.: startle eye blink; im Folgenden allgemein: Schreckreaktion) als erste und stabilste Komponente, um diesen Schutzreflex systematisch zu erfassen \parencite{LANG1990}. Die Lidschlusskomponente beschreibt das reflexartige Zusammenkneifen der Augen infolge eines meist akustischen Schreckreizes. Wird dieser im Labor präsentiert, kann die Muskelkontraktion des Musculus Orbicularis Oculi über ein Elektromyogramm (EMG) aufgezeichnet werden. Die Elektroden hierfür werden zumeist auf der Hautoberfläche unterhalb des Auges angebracht. Der Schreckreiz ist meist ein \SI{50}{\milli\second} kurzes, $90$ bis \SI{105}{\decibel} lautes, weißes Rauschen und nach seiner Vergabe erfolgt die Schreckreaktion meist innerhalb von \SI{150}{\milli\second} \parencite{BLUMENTHAL2005}.
		Dadurch, dass diese Reaktion automatisch ausgelöst wird und sich damit der willentlichen Kontrolle entzieht, ist sie resistent gegenüber Effekten kognitiver Verarbeitung und Antworttendenzen, wie sie z.\,B. in verbalen Berichten auftreten können \parencite{LIPP2006, GRILLON2003}. Das ist methodisch ein Vorteil, da andere physiologische Maße wie die Herzrate und die Hautleitfähigkeit durchaus – wenn auch in begrenztem Maß -- aktiv durch die Versuchspersonen beeinflusst werden können \parencite[z.\,B. mit Beeinflussung der Atmung;][]{LONSDORF2017fc}.
		
		Bei Verwendung in Furchtkonditionierungsstudien ist ein wesentlicher methodischer Vorteil der Schreckreaktion, dass sie circa um das $50$-fache schneller nach Reizdarbietung auftritt als die Hautleitwertreaktion. %(min. $20$\,ms versus min. $1$\,s)
		Außerdem weist sie eine kürzere Refraktärzeit auf. Durch diesen zeitlichen Vorteil kann die Schreckreaktion zu jeder Zeit während des Furchtakquisitionstrainings ausgelöst werden und wichtige zeitlich kontingente Einblicke liefern \parencite{LIPP2006}. Im Gegensatz zur SCR ist es außerdem möglich, Reaktionen innerhalb der ITI zu erheben und damit eine Basisreaktion unabhängig von der CS-Präsentation zu definieren. Die Erfassung der Schreckreaktion als Furchtmaß benötigt allerdings eine Habituationsphase, in der der Schreckreiz zwei bis zehn Mal präsentiert wird. Damit soll gewährleistet werden, dass die ersten starken Reaktionen auf den Schreckreiz den Verlauf der CR in Reaktion auf einen Stimulus weniger beeinflussen \parencite{BLUMENTHAL2005}. Eine Phase wie diese ist bei Erhebung der SCR nicht nötig. 
		
	\subsection{Verwendung als Indikator}				\label{startleinference}
		
		In den allermeisten Fällen, in denen die Schreckreaktion experimentell erhoben wird, geht es nicht primär um den motorischen Reflex als solchen, sondern um die psychologischen Prozesse, mit denen er in Verbindung gebracht wird. Die Relevanz der Schreckreaktion für die Furchtkonditionierungsforschung begründet sich durch ihre relative Spezifität zur Erfassung von Furcht. 
		\textcite{LANG1990} zeigten, dass die Schreckreaktion durch Affekt moduliert wird. Während angenehme Reize die Schreckreaktion hemmen, wird sie durch unangenehme Stimuli potenziert. Dieses Phänomen ist im Kontext der Furchtkonditionierung als Furchtpotenzierung der Schreckreaktion (FPS) bekannt: Ein bereits konditionierter Stimulus (CS+) löst eine stärkere Schreckreaktion aus als ein nicht-verstärkter CS-- \parencite{BROWN1951, HAMM1996, HAMM2005, BRADLEY2005}. Die FPS tritt unabhängig von der Neuheit der Reize auf. Zwar nimmt die Reaktion in absoluten Werten bei mehreren Trials mit der Zeit ab, der Effekt der emotionalen Modulation bleibt jedoch robust bestehen \parencite{BRADLEY2000}. Zudem zeigten \textcite{CUTHBERT1996}, dass Erregung die Schreckreaktion auf aversive Reize zusätzlich moduliert. Die Potenzierung der Schreckreaktion zeigt sich bei der Präsentation von hoch erregenden und negativ valenten Stimuli, allerdings nicht bei wenig erregenden und trotzdem negativ valenten Reizen \parencite{CUTHBERT1996}. Ein daraus resultierender Vorteil ist, dass die Schreckreaktion Furchtlernen verhältnismäßig spezifisch erfassen kann. Im Vergleich mit der Hautleitwertreaktion, die eher valenzunspezifische Erregung reflektiert, wird der Schreckreaktion die Rolle als spezifischerer Furchtindikator zugedacht \parencite{HAMM1996}. 
		
		Die neuroanatomische Basis wurde vielfach in tierexperimentellen Studien untersucht. Zahlreiche Studien, u.a. von Davis und Kolleg*innen, lieferten übereinstimmende Evidenz dafür, dass die Potenzierung der Schreckreaktion über die Amygdala vermittelt wird \parencite[für eine Übersichtsarbeit siehe][]{DAVIS2006}. Die Amygdala ist eine limbische Struktur im medialen Temporallappen und gilt als eine zentrale Schaltstelle des Defensivsystems \parencite{HAMM2006, DAVIS1992}. Mit ihren efferenten Projektionen initiiert und mediiert die Amygdala Furchtsymptome wie körperliche Erregung oder protektive motorische Reflexe, wozu auch die Schreckreaktion gehört. 
		Bei Läsionen des zentralen Kerns der Amygdala findet die Schreckreaktion auf einen Schreckreiz durchaus statt, allerdings wird sie bei Furchtreizen nicht mehr potenziert \parencite{HITCHCOCK1987, HITCHCOCK1986}. Das macht die Schreckreaktion zu einem validen Indikator für die Aktivierung des Defensivsystems.
		
		Ein methodischer und theoretischer Nachteil im Vergleich zur SCR ergibt sich durch die Notwendigkeit eines zusätzlichen Schreckreizes zum Auslösen der Schreckreaktion. Es ist bereits gezeigt worden, dass die physikalischen Eigenschaften des Schreckreizes die Schreckreaktion in Intensität und Onset modulieren \parencite{BLUMENTHAL2005}. Außerdem wird der Schreckreiz von Versuchspersonen ebenfalls häufig als aversiv wahrgenommen. Ein potentieller Einfluss auf den emotionalen Zustand der Person und demnach auch auf die Intensität des Schreckreflexes kann nicht ausgeschlossen werden \parencite{LIPP2006}. \textcite{SJOUWERMAN2016} konnten außerdem zeigen, dass die Inklusion von Schreckreizen in einer Furchtkonditionierungsstudie die Furchtakquisition verzögerte (gemessen anhand der SCR). Die Wahl der abhängigen Variable(n) hat demnach durchaus einen nicht zu unterschätzenden Einfluss auf den Lernprozess.
		
		
%%%%%%%%%%%%%%%%%%%%%%%%%%%%%%%%%%%%%%%%%%%%%
%%%%%%%%%%%%%%MEHRERE OUTCOMES%%%%%%%%%%%%%%%%%%%
%%%%%%%%%%%%%%%%%%%%%%%%%%%%%%%%%%%%%%%%%%%%%
			
\section{Verwendung mehrerer Indikatoren}			\label{outcome}

	Furchtlernen gilt als Konstrukt mit einem komplexen Aktivierungsnetzwerk. Die eingesetzten Indikatoren können nicht distinkt in die spezifische Erfassung von Kontingenzwissen, konditionierter Reaktion, Aufmerksamkeitsverarbeitung oder emotionalem Zustand eingeordnet werden \parencite{LIPP2006}. Es hat sich während der letzten Forschungsjahrzehnte etabliert, mehrere Indikatoren gleicher oder unterschiedlicher Reaktionsebenen zur Erfassung von Furchtlernen zu verwenden, um ein vollständigeres Bild der Prozesse zu erhalten. Nach \textcite{LIPP2006} ging diese wichtige Änderung mit der Realisierung einher, dass einzelne Maße unterschiedliche Spezifitäten aufweisen. Verbale Maße sind oft durch Antwort- und Erinnerungstendenzen verfälscht. Physiologische Maße reflektieren nicht nur Furcht, sondern werden ebenfalls durch Prozesse wie Aufmerksamkeit oder Kontexteffekte beeinflusst. Es wird auch in der Empirie deutlich, dass Indikatoren verschiedener Ebenen oft nur in geringem Maße kovariieren \parencite{BRADLEY2000, LIPP2006}. 
	Ein verwandtes Problem beim Verwenden mehrerer Maße ist, dass sich die Indikatoren auch in ihrer Sensitivität unterscheiden. Der direkte Vergleich von unterschiedlichen CR, vor allem über verschiedene Studien hinweg, gestaltet sich als schwieriges Unterfangen. 
	Zusammenfassend bedeutet das: Ein einzelnes Maß kann viele Komponenten abbilden, währenddessen sich die gewünschten Effekte in mehreren Maßen gleichzeitig widerspiegeln können.
	Die Integration mehrerer Indikatoren genauso wie das Wissen um ihre Interdependenzen sind entscheidende Bausteine der aktuellen und zukünftigen Furchtkonditionierungsforschung. 
	
	Die steigende Komplexität, die die Wahl der abhängigen Variablen mit sich bringt, ist auch zunehmend Gegenstand der Forschungsliteratur. Neben den bereits zitierten Empfehlungen und Richtlinien von \textcite{LONSDORF2017fc} haben auch \textcite{OJALA2020} erst kürzlich eine Übersichtsarbeit über verschiedene Arten konditionierter Reaktionen veröffentlicht. Darin beurteilten sie verschiedene CR unter anderem nach ihrer Qualität, Inferenzen auf gebildete CS-US-Assoziationen zu ermöglichen. Eine Studie von \textcite{LEUCHS2019} verglich die Indikatoren SCR und Schreckreaktion mit der an Aufmerksamkeit gewinnenden Pupillendilatation als CR einer Furchtkonditionierung. Die Akquisitionsdaten zeigen signifikante, aber eher geringe Korrelationen zwischen allen drei Maßen (trialweise gemittelt; für SCR und FPS $r=.16$, $p<.001$, \nptextcite{LEUCHS2019}). Sowohl die SCR als auch die FPS unterlagen einer starken Habituation, diskriminierten aber beide signifikant zwischen CS+ und CS-- \parencite{LEUCHS2019}.
	Relevant für den Einsatz mehrerer Reaktionsmaße ist auch, dass sich verschiedene CR durchaus gegenseitig beeinflussen \parencite{LONSDORF2017fc}. Das liegt nicht nur daran, dass Trials und ITI in Rücksicht auf die Onset- und Refraktärzeiten der Hautleitwertreaktion zeitlich koordiniert werden müssen, sondern auch an der Wechselwirkung von CR an sich. Dazu gehört zum Beispiel die bereits erwähnte Studie von \textcite{SJOUWERMAN2016}, bei der das Verwenden von akustischen Schreckreizen das Furchtlernen, gezeigt an der SCR, verzögert. Ein weiterer, wichtiger Aspekt ist die umstrittene Annahme, dass unterschiedliche Indikatoren verschiedene inhaltliche Prozesse des Furchtlernens abbilden. Im Folgenden wird dieser Disput in der Forschung näher erläutert.


%%%%%%%%%%%%%%%%%%%%%%%%%%%%%%%%%%%%%%%%%%%%%
%%%%%%%%%%%%%%LERNTHEORETISCH%%%%%%%%%%%%%%%%%%%%%
%%%%%%%%%%%%%%%%%%%%%%%%%%%%%%%%%%%%%%%%%%%%%
	
\section{Lerntheoretische Perspektiven}			\label{processes}

	\subsection{Ein oder zwei Systeme?}			\label{singledual}

		Ursprünglich wurde der Konditionierungsprozess als simple, beinahe mechanische Bildung von Assoziationen unabhängig von kognitiven Mechanismen verstanden \parencite{LOVIBOND2006}. Es ist mittlerweile größtenteils unbestritten, dass auch kognitive Mechanismen (wie die Abhängigkeit des Lernens von Kontingenz und Kontiguität, \nptextcite{RESCORLA1968}) bei der Erklärung von Furchtlernphänomenen eine nicht zu unterschätzende Rolle spielen. In diesem Zusammenhang werden häufig zwei aktive Systeme angenommen, sobald Menschen Reizkontingenzen beim klassischen Konditionieren ausgesetzt sind: zum einen das automatische, reflexive System und zum anderen das propositionale, kognitive System \parencite{LOVIBOND2006}. Diese Dualitätsannahme taucht in verschiedenen Ansätzen zu Lern- und Gedächtnisprozessen mit jeweils kleinen Abweichungen immer wieder auf. Begrifflich werden theoretische Differenzierungen zwischen dem expliziten und dem impliziten, dem bewussten und dem unbewussten oder dem kontrollierten und dem automatischen Lernen vorgenommen. Die Idee von zwei \textit{Ebenen des Lernens} beim menschlichen Assoziationslernen reflektiert sich in den verschiedenen dualen Lerntheorien. Doch auch wenn der lernbasierte Ansatz und die substanzielle Bedeutung von Kognitionen in Lernprozessen allgemein akzeptiert bleibt, ist das Vorhandensein dualer oder mehrerer Prozesse an sich ein umstrittenes Thema.

		In der Furchtkonditionierungsforschung beschäftigt sich der Diskurs damit, ob überhaupt distinkte Prozesse am Klassischen Konditionieren beteiligt sind. Ein Fokus liegt hier auf dem Konflikt um die Notwendigkeit des Bewusstseins für das Zeigen einer konditionierten Reaktion \parencite[für systematische Übersichtsarbeiten siehe][]{LOVIBOND2002, MERTENS2020}. Es wird unter anderem darüber disputiert, ob und zu welchem Grad die Konditionierung beim Menschen von einem bewussten Verständnis der gelernten Kontingenz abhängt und welche Operationalisierung für Bewusstsein hierfür herangezogen werden kann. Maßgeblich wird der Konflikt von zwei Polen bestimmt. Die \textit{Zwei-Prozess}-Theorie vertritt die Auffassung von zwei unabhängigen Prozessen des Lernens \parencite{HAMM1996, HAMM2005, OEHMANN2001, WEIKE2007}. Zum einen führt ein propositionaler, kognitiver Lernvorgang zu verbalisierbarem Kontingenzbewusstsein. Zum anderen wird auf einer nicht-propositionalen, impliziten Ebene eine autonome CR wie die Schreckreaktion produziert. 
		Die \textit{Ein-Prozess}-Theorie steht dem gegenüber. Vertreter*innen dieser Annahme kritisieren, dass in Studien zum Klassischen Konditionieren beim Menschen kein hinreichender Nachweis für die Notwendigkeit von Bewusstsein -- und damit für das Vorhandensein zweier Ebenen -- erbracht wurde. Der Ein-Prozess-Ansatz geht von nur einer Ebene aus, die einen propositionalen Lernprozess beschreibt \parencite{LOVIBOND2002, DEHOUWER2014, DEHOUWER2009, MITCHELL2009, MERTENS2020}. In seiner stärksten Form wird angenommen, dass Kontingenzbewusstsein eine Voraussetzung für das Auftreten einer CR ist. Personen müssen demnach in der Lage sein, explizit die Beziehung zwischen CS und US zu beschreiben, um konditionierte Reaktionen zu zeigen. In einer schwächeren Form der Ein-Prozess-Theorie sind sowohl Kontingenzbewusstsein als auch CR ebenbürtige Endpunkte des propositionalen Lernprozesses, die allerdings stark miteinander korrespondieren. 

		Die Forschungsergebnisse zu dem Disput sind bis heute ambig. Unter anderem kritisierten \textcite{LOVIBOND2002} die Operationalisierung und Erfassung von Kontingenzbewusstsein und bemängelten, dass kein hinreichender Nachweis für einen Dualismus oder ein Konditionieren ohne Bewusstsein erbracht worden sei. Außerdem erweise sich die unterschiedliche Methodik in der Durchführung von Konditionierungsstudien als Schwierigkeit für die Interpretation und Vergleichbarkeit unterschiedlicher Forschungsergebnisse. In ihrer aktuellen Metaanalyse verzeichneten \textcite{MERTENS2020}, dass $39$ von $41$ inkludierten Studien mindestens einem methodischen Problem unterlagen. Dazu gehören unter anderem die Erfassung von Kontingenzbewusstsein und Entscheidungsfreiheiten der Wissenschaftler*innen in Bezug auf Ausschlusskriterien und Kennwertbestimmung.
		
		Auch wenn grundlegend relevant für die aktuelle Forschung, so wird das Konstrukt \textit{Kontingenzbewusstsein} als solches und der Disput um seine Operationalisierung nicht näher in dieser Arbeit betrachtet.
		Interessierte Leser*innen seien zum Beispiel an die Arbeiten von \textcite{LOVIBOND2002}, \textcite{WIENS2002} oder \textcite{MERTENS2020} verwiesen. Im nächsten Abschnitt folgt ein Überblick über Forschungsergebnisse, bei denen Hautleitwert- und Schreckreaktionen als Indikatoren verschiedener Lernprozesse herangezogen wurden. 
		Dies dient als Grundlage für die Herleitung der Fragestellung und Analyse in dieser Arbeit.

	\subsection{Schreck- und Hautleitwertreaktion als Indikatoren unterschiedlicher Lernebenen}	\label{dissociation}
		
		Insbesondere im Rahmen der Zwei-Prozess-Theorie werden Schreck- und Hautleitwertreaktionen als Indikatoren verschiedener Lernebenen herangezogen \parencite{HAMM1996, SEVENSTER2014}. Es wird angenommen, dass die FPS das implizite und emotionale Furchtlernen reflektiert und unabhängiger von kognitivem Wissen über die Stimulusrelationen auftritt \parencite[z.\,B.][]{SEVENSTER2014, HAMM2005}. Dagegen wird die differentielle SCR grundlegend mit Kontingenzlernen und dem US-Erwartungslernen in Verbindung gebracht \parencite[z.\,B.][]{HAMM2005, WEIKE2007}. US-Erwartung beschreibt, wie groß die Erwartung der Versuchspersonen an einen US ist, nachdem ein bestimmter CS präsentiert wurde. Allerdings ist der Zusammenhang zwischen SCR und Kontingenzbewusstsein umstritten.
		Es herrscht konkurrierende Evidenz darüber, ob das Zeigen einer differentiellen SCR vom Kontingenzbewusstsein der Person abhängt \parencite{DAWSON1973, HAMM1996, PURKIS2001, WEIKE2007} oder nicht \parencite{KNIGHT2003, SCHULTZ2010}.
		
		Ein Beispiel für die Dissoziation zwischen SCR und Schreckreaktion in einem differentiellen Furchtkonditionierungsparadigma lieferten \textcite{SEVENSTER2014}. Sie reproduzierten eine Studie von \textcite{SCHULTZ2010}, bei der eine Gruppe auf ein einfaches und eine andere auf ein schwierig zu diskriminierendes Paar von CS+ und CS-- konditioniert wurde. \textcite{SCHULTZ2010} fanden eine differentielle US-Erwartung ursprünglich nur in der leichten Bedingung, während beide Gruppen eine differentielle SCR zeigten. Daraus schlussfolgerten sie, dass die differentielle SCR nicht durch Kontingenzbewusstsein beeinflusst wird \parencite{SCHULTZ2010}. In ihrer Replikationsstudie verwendeten \textcite{SEVENSTER2014} dasselbe Set an Stimuli und erfassten zusätzlich die Schreckreaktion als Indikator. Ihre Ergebnisse zeigten, dass die differentielle SCR nur bei den Versuchspersonen auftrat, die deklaratives Wissen über die CS-US-Kontingenz aufgebaut hatten (erfasst über Befragungen zur US-Erwartung während des Trainings). Eine differentielle Schreckreaktion zeigte sich unabhängig vom Kontingenzwissen. Diese Daten unterstützen die Zwei-Prozess-Theorie und lassen vermuten, dass die Hautleitwertreaktion im Gegensatz zur Schreckreaktion mit bewusstem Lernen über die Stimuluskontingenzen zusammenhängt. 
		Die Ergebnisse der Studie von \textcite{WEIKE2007} deuten zudem auf eine Abhängigkeit dieser potentiellen Dissoziation von der Methodik der CS-US-Paarungen hin. In einem Vergleich zwischen Spuren- und Verzögerungskonditionierung zeigte sich die differentielle SCR in beiden Gruppen nur bei Personen, die Kontingenzbewusstsein aufwiesen (erfasst über ein Interview nach dem Training). Die konditionierte FPS allerdings trat in der Gruppe mit Verzögerungskonditionierung unabhängig vom Bewusstsein auf, während sie bei der Gruppe mit Spurenkonditionierung nur bei Personen mit Kontingenzbewusstsein zu finden war. 
		Daraus lässt sich schließen, dass auch methodische und zeitliche Aspekte des Furchtakquisitionstrainings einen Einfluss darauf haben, inwiefern Schreck- und Hautleitwertreaktionen als Indikatoren zweier distinkter Lernprozesse herangezogen werden können. Limitiert wird die Interpretierbarkeit dieses Ergebnisses durch die Erfassung von Kontingenzbewusstsein nach dem Akquisitionstraining, wodurch Gedächtniseffekte nicht ausgeschlossen werden können \parencite{MERTENS2020}. 
		
		Auf der anderen Seite gibt es auch Befunde, die keine Nachweise für Furchtkonditionierung der Schreckreaktion in Abwesenheit von Kontingenzbewusstsein finden und demnach keine Inferenz auf distinkte Prozesse zulassen. In einer Furchtakquisitionsstudie mit Verzögerungskonditionierung von \textcite{PURKIS2001} zeigten sich differentielle Hautleitwertreaktionen nur für die Versuchspersonen, die Kontingenzbewusstsein aufwiesen. Das gleiche Muster zeichnete sich allerdings auch für die Schreckreaktionen ab, was nahelegt, dass die FPS abhängig vom Kontingenzbewusstsein der Versuchspersonen ist.
		Ähnliche Ergebnisse erbrachte die Studie von \textcite{GRILLON2002}, bei der ebenfalls nur Versuchspersonen mit Kontingenzbewusstsein differentielle SCR und FPS zeigten.
		Diese Daten sind aufgrund der fehlenden Diskrepanz zwischen propositionalem Wissen und dem Zeigen einer differentiellen CR konsistent mit dem Ein-Prozess-Ansatz. 
		
		
		Abseits der hauptsächlich peripherphysiologischen Forschung wurde die Dualitätsannahme im Furchtlernen auch auf neurobiologischer Basis in Tierstudien untersucht. In einer der bekanntesten Theorien zur Rolle der Amygdala im Furchtnetzwerk postuliert \textcite{LEDOUX2003} zwei distinkte Routen der Furchtaktivierung. Auf der sogenannten automatischen \textit{low road} wird der furchtauslösende Input über den Thalamus (Kernstruktur im Zwischenhirn) direkt an die Amygdala geleitet, ohne kortikal verarbeitet zu werden. Die Amygdala wiederum initiiert eine schnelle präventive Reflexreaktion, die sich zum Beispiel im Zusammenkneifen der Augen zeigt (Schreckreaktion). Die \textit{high road} hingegen ist langsamer: Der visuelle Input wird über den Thalamus an den Kortex geleitet und dort analysiert und eingeordnet. Während \textcite{LEDOUX1996} argumentiert, dass sich der Amygdala-Schaltkreis weitestgehend kognitiver Kontrolle entzieht, werden im Weg über den Kortex kognitive Prozesse mit einbezogen.
		Diese neurobiologische Unterscheidung legt nahe, dass die Aktivierung des Defensivsystems, vermittelt über die Amygdala, vom Erwerb deklarativen Wissens auf kognitiver Ebene differenziert werden kann \parencite{OEHMANN2001}. Dissoziative Ergebnisse, in denen sich Schreckreaktionen, aber nicht Hautleitwertreaktionen unabhängig vom propositionalen Kontingenzwissen zeigen, sind mit dieser Theorie von \textcite{LEDOUX2003} vereinbar.
		In einer Bildgebungsstudie beim Menschen konnte bei Versuchspersonen, die kein Kontingenzbewusstsein zeigten, auch eine erhöhte Aktivität der Amygdala auf den CS+ nachgewiesen werden \parencite{TABBERT2006}. Zudem erfassten \textcite{TABBERT2006} Hautleitwertreaktionen, die nur in der Gruppe mit Kontingenzbewusstsein signifikante Unterschiede zwischen CS+ und CS-- aufwiesen. Diese Ergebnisse unterstützen die Annahme einer Aktivierung des Furchtnetzwerks über die Amygdala in Abwesenheit (und damit unabhängig) von deklarativem Kontingenzwissen. Auch in dieser Studie wurde Kontingenzbewusstsein über einen Fragebogen nach dem eigentlichen Akquisitionstraining erfasst, was als Limitation zu betrachten ist \parencite{MERTENS2020}. %ggf. Lindner, 2015
		
		Die Forschungsergebnisse und ihre Diskrepanzen zeigen, dass man noch ein gutes Stück davon entfernt ist, die genauen Mechanismen des Furchtlernens zu verstehen. 
		Die Argumentation der Zwei-Prozess-Theorie führt einen zu der grundlegenden Unterscheidung zwischen dem emotionalen und dem kognitiven Lernen \parencite{OEHMANN2001, HAMM2005}. Ersteres beschreibt die Aktivierung des Defensivsystems als Furchtreaktion auf den CS. Diese Aktivierung kann über die direkte Verbindung zwischen Thalamus und Amygdala erklärt werden \parencite{LEDOUX2003}. Die Potenzierung der Schreckreaktion als reliabler und relativ spezifischer Furchtindikator wird wiederum über die Amygdala vermittelt (\nptextcite{DAVIS2006}; Abschnitt \ref{startleinference}). Daneben beschreibt das kognitive Lernen das propositionale Erlernen des Zusammenhangs zwischen dem CS und dem US. Dieser deklarative Wissenserwerb wird eher mit einer differentiellen SCR assoziiert. Eine höhere Hautleitwertreaktion erfolgt auf den Stimuli, der ein signifikantes Ereignis vorhersagt, unabhängig davon, ob es aversiv ist oder nicht \parencite{HAMM1996, HAMM2005}. Dies deckt sich mit den Forschungsergebnissen, die die SCR mit Orientierungsreaktionen auf neue und signifikante Stimuli in Verbindung bringen (\nptextcite{SIDDLE1991}; Abschnitt \ref{scrinference}). 	
		Eine solch klare Unterscheidung zeigt sich jedoch nicht immer in den Forschungsergebnissen. Es bedarf in jedem Fall weiterer empirischer Forschung, um die Beziehung zwischen verschiedenen Indikatoren des Furchtlernens besser nachvollziehen zu können. Obwohl es viele Studien zu Schreck- und Hautleitwertreaktionen im Furchtlernkontext gibt, existiert dennoch wenig Literatur, die beide Maße direkt miteinander in Beziehung setzt \parencite{OJALA2020}. Erst kürzlich haben \textcite{CONSTANTINOU2021} in einer Furchtkonditionierungsstudie u.\,a. Mehrebenenmodelle verwendet, um SCR-Reaktionen mit US-Erwartungsratings in Verbindung zu setzen. %Ihre Ergebnisse unterstützen die Annahme, dass SCR-Reaktionen kognitive Lernprozesse reflektieren.
		Um zu diesem Forschungsgebiet beizutragen und der Frage nachzugehen, ob sich mögliche distinkte Lernprozesse auf der physiologischen Reaktionsebene zeigen, strebe ich in dieser Arbeit einen Vergleich der beiden Indikatoren Schreck- und Hautleitwertreaktionen an.
		

%%%%%%%%%%%%%LEITFADEN BACHELORARBEIT
% Nach einer ausführlichen Literaturrecherche unter Zuhilfenahme relevanter Datenbanken wie bspw. PsycINFO werden im Theorieteil der Arbeit (a) die relevanten theoretischen Konstrukte definiert, (b) der aktuelle Forschungsstand dargestellt und (c) offene Forschungsfragen identifiziert. Dabei ist immer darauf zu achten, ob die dargestellten Informationen hinreichend und notwendig sind um die Fragestellungen bzw. Hypothesen zu begründen, d.h. hier ist die Auswahl der relevanten Literatur ein entscheidendes Bewertungskriterium. Relevant sind neben den aktuellsten Entwicklungen in einem Forschungsbereich in der Regel besonders einflussreiche, d.h. viel zitierte empirische und theoretische Arbeiten sowie aktuelle Übersichtsarbeiten und Meta-Analysen. Des Weiteren wird hier auch die Integration sowie kritische Bewertung der zitierten Literatur geprüft.
% 
%\textbf{Leitfragen:} Wurden alle zentralen Begriffe eingeführt und klar definiert? Sind die berücksichtigten Arbeiten für das Thema repräsentativ und relevant? Sind die Inhalte gut gegliedert und verständlich dargestellt? Wird die berücksichtigte Literatur aufeinander bezogen (integriert) und kritisch reflektiert?
			
			
