
%%%%%%%%%%%%%%%%%%%%%%%%%%%%%%%%%%%%%%%%%%%%%
%%%%%%%%%%%%%%% ABSTRACT %%%%%%%%%%%%%%%%%%%%
%%%%%%%%%%%%%%%%%%%%%%%%%%%%%%%%%%%%%%%%%%%%%
%aktuell 249 Wörter

\noindent 
%motivation and problem statement
 	In der Furchtkonditionierungsforschung sind Schreck- und Hautleitwertreaktionen (SCR) zwei der verbreitetsten Indikatoren, um  humane Furchtlernprozesse zu untersuchen. Allerdings herrscht Uneinigkeit darüber, inwiefern sie distinkte Lernprozesse widerspiegeln. Die Zwei-Prozess-Theorie nimmt an, dass differentielle Schreckreaktionen eine automatische Ebene des Furchtlernens indizieren, während differentielle SCR Kontingenzwissen auf kognitiver Ebene reflektieren.
% approach
	Diese Arbeit untersucht explorativ, ob sich in einer Furchtakquisition Unterschiede in Verläufen und Diskrimination Konditionierter Stimuli (CS) zwischen SCR und Schreckreaktionen abzeichnen, wenn beide statistisch gleichbehandelt in einem multivariaten Modell ausgewertet werden. 
	%Dafür werden Teildaten eines Großprojekts untersucht. 
	Achtunddreißig Versuchspersonen durchlaufen ein differentielles %, partiell verstärkten 
	Furchtakquisitionstraining, in dessen Verlauf ihnen zwei geometrische Reize mehrfach präsentiert werden. Einem folgt mit \SI{50}{\percent} Wahrscheinlichkeit ein elektrotaktiler Reiz. Nach der Hälfte des Trainings werden Versuchspersonen über die Reizkontingenzen aufgeklärt.
	Auswertungen erfolgen durch eine multivariate Wachstumskurvenanalyse (als Mehrebenenmodell). Mit dieser können Veränderungen zweier Variablen über die Zeit hinweg und die Variabilität zwischen Versuchspersonen modelliert werden. 		
% results
	Wichtigste Ergebnisse zeigen einen Interaktionseffekt zwischen CS und Zeit für die SCR -- nicht aber für die Schreckreaktionen --, der substanziell zwischen Versuchspersonen variiert. Mittelwertvergleiche und Visualisierungen verdeutlichen, dass eine Differenzierung zwischen den CS wenn, dann erst nach der Instruktion, auftritt.	
% conclusion
	Die gefundenen Effekte lassen sich schlussfolgernd nicht auf Furchtlernprozesse, sondern auf einen Instruktionseffekt zurückführen. Inferenzen auf die theoretische Indikation mehrerer Lernebenen lassen sich nicht ziehen.		
% future directions
	Nichtsdestotrotz werden wesentliche Vorteile, aber auch Hürden beim Einsatz von multivariaten Mehrebenenmodellen erläutert. 
	Dieser statistische Ansatz eröffnet neue Möglichkeiten zur Untersuchung und Integration mehrerer Furchtindikatoren und birgt das Potential, einen Beitrag zur Entschlüsselung der Mechanismen potentiell distinkter Furchtlernprozesse zu leisten.
	
	
\vspace*{0.5cm}
\noindent \textit{Schlagwörter:} Furchtakquisition, Hautleitwertreaktion, Schreckreaktion, Furchtlernen, Mehrebenenmodelle, Multivariate Verfahren

%Background, Methods, Results, Conclusions 

%Background & objectives, Methods, Results, Limitations, Conclusions

%Fragestellung, Hypothese, Merkmale der Versuchspersonen (z. B. Anzahl, Alter, Geschlecht), Methode, zentrale Ergebnisse sowie Schlussfolgerungen für die Hypothese.

%Fragestellung, Methode, Ergebnisse, Schlussfolgerungen.


%Beispiel Manon: Transkutane Vagus-Nerv-Stimulation (t-VNS) ist ein neues, nichtinvasives Neurostimulationsverfahren um den zervikalen Vagus-Nerv zu stimulieren, welches, vermutlich über die Aktivierung des Locus-Coeruleus-Noradrenalin-Systems, kognitive und affektive Prozesse beeinflusst. Mittels einem einfachblinden, randomisierten Studiendesigns untersuchte die folgende Studie die Effekte von t-VNS auf die Leistung von 61 gesunden Probanden bei einer lexikalischen Entscheidungsaufgabe und einem darauffolgenden Rekognitionstest. Hierbei erhielten Probanden entweder diskontinuierliche t-VNS (aktive Stimulation) oder diskontinuierliche Sham-Stimulation (Kontroll-Stimulation), während ihnen Wörter und Pseudowörter, die sich in ihrer emotionalen Valenz und in ihrer Abstraktheit unterschieden, präsentiert wurden. Des Weiteren wurde der Speichel-Alpha-Amylase (sAA) Spiegel als Biomarker für noradrenerge Aktivität vor und nach der Stimulation erfasst. Die Ergebnisse ergaben keine Hinweise auf Veränderungen des sAA Spiegels nach t-VNS. Dennoch konnte ein Interaktionseffekt von t-VNS und Wortvalenz im Rekognitionstest gezeigt werden, welcher darauf hinweist, dass t-VNS unterschiedliche Einflüsse auf die Reaktionszeit auf positive und negative Wörter hatte. Zudem waren sich Probanden aus der aktiven Stimulationsbedingung sicherer in ihren Entscheidungen an welche Wörter sie sich erinnerten und an welche nicht. Es konnte kein Einfluss von t-VNS auf Wortabstraktheit gefunden werden. Schlussfolgernd ist es unklar, ob das angewendete Stimulationsprotokoll zu einer erfolgreichen Stimulation des Vagus-Nervs geführt hat. Diese Ergebnisse betonen die Notwendigkeit, verlässliche Indikatoren für noradrenerge Aktivierung und optimale Stimulationsprotokolle weiter zu erforschen, um die Relevanz von t-VNS für potentielle Forschung (z.B. zu Gedächtnis, Emotionen) und für klinische Anwendungen (z.B. als therapeutisches Mittel) zu klären. 


%%%%%%%%%%%%%%%LEITFADEN BA
%Vollständigkeit: Das Abstract soll alle erforderlichen Informationen enthalten: Fragestellung, Hypothese, Merkmale der Versuchspersonen (z. B. Anzahl, Alter, Geschlecht), Methode, zentrale Ergebnisse sowie Schlussfolgerungen für die Hypothese.\\
%Genauigkeit: Inhaltliche Schwerpunkte und Terminologie des Textes sollen im Abstract beibehalten werden. Das Abstract darf keine Informationen enthalten, die im Manuskripttext nicht genannt werden.\\ 
%Objektivität: Das Abstract soll den Inhalt des Texts ohne Wertung wiedergeben.\\
%Kürze: Das Abstract soll so kurz wie möglich sein (nicht mehr als 250 Wörter); Unwesentliches, Wiederholungen sowie redundante Redewendungen sollen vermieden werden. Die APA empfiehlt weiterhin für Abstracts den Gebrauch von Ziffern statt verbaler Zahlbezeichnungen (außer am Satzanfang) und die Verwendung gebräuchlicher Abkürzungen (z. B. vs. statt versus), sowie der Verbform des Aktivs anstelle des Passivs (ohne die Personalpronomen ich und wir). Nichtlateinische Schriftzeichen, Symbole und Formeln sollten, wenn möglich, vermieden werden (wegen der Speicherung in Datenbanken mit begrenztem Zeichensatz).\\ 
%Verständlichkeit: Das Abstract soll klar und verständlich formuliert sein. Verschachtelte Sätze sollen vermieden werden; die wesentlichen Begriffe sollen in den Formulierungen deutlich hervortreten. Das Abstract muss auch ohne Fachkenntnisse, die über eine durchschnittliche psychologische Vorbildung hinausgehen, verständlich sein. Nicht allgemeingebräuchliche Abkürzungen müssen bei der ersten Nennung erläutert werden. Testverfahren sollen (bei der ersten Nennung) in ausgeschriebener Form angegeben werden.\\