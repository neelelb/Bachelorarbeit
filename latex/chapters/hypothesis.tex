
%\section{Ziel der Arbeit}								\label{aim}

	Anhand der Literatur wird deutlich, dass sowohl Schreck- als auch Hautleitwertreaktionen viel verwendete Indikatoren für das Furchtlernen sind. Ob und inwiefern die beiden Maße tatsächlich als Indikatoren verschiedener Lernprozesse herangezogen werden können, bleibt allerdings größtenteils ungeklärt. 
	Um einen anderen Blickwinkel auf diese Diskordanz zu erhalten, umfasst diese Arbeit einen direkten explorativen Vergleich beider Maße als abhängige Variablen in einer Furchtakquisition. 
	Wenn, wie die Zwei-Prozess-Theorie nahelegt, die beiden physiologischen Maße durch verschiedene kognitiv-affektive Prozesse moduliert werden, ließe sich vermuten, dass sich dies in unterschiedlichen zeitlichen Antwortmustern während des Furchtlernens widerspiegelt. %Im Hinblick darauf, ob sich mögliche distinkte Lernprozesse auf Reaktionsebene zeigen, strebe ich in dieser Arbeit einen Vergleich der beiden Indikatoren Schreck- und Hautleitwertreaktionen an.

	Um die Beziehung zwischen den beiden Indikatoren besser zu verstehen, wird in einem explorativen Ansatz beleuchtet, ob die Schreck- und Hautleitwertreaktionen in der Furchtakquisitionsphase unterschiedliche Lernprozesse auf der Signalebene abbilden. Die Aufmerksamkeit liegt auf der Frage, ob sich unterschiedliche Lernverläufe in Hautleitwert- und Schreckreaktion abzeichnen, wenn beide Maße im Modell gleich behandelt werden. Mit anderen Worten: Wären verschiedene Lernsysteme -- sofern sie existieren -- auf Reaktionsebene durch den Lernverlauf der beiden Maße unterscheidbar?
	

	Für diesen Ansatz werden die Daten eines differentiellen, partiell verstärkten, verzögerten Furchtakquisitionstraining erhoben und analysiert.
	Für den Vergleich der beiden abhängigen Variablen wird eine multivariate Wachstumskurvenanalyse im Rahmen einer Mehrebenenmodellierung durchgeführt.
	Innerhalb dieses multivariaten Modells werden die Veränderungen der beiden Reaktionsmaße über die Trials hinweg als Wachstumskurven modelliert und der Einfluss des Stimulustyps untersucht. %und ggf. der Trialreihenfolge %Die Trialreihenfolge beschreibt, mit welchem Stimulustyp (CS+, CS--) das Akquisitionstraining beginnt und ob der erste CS+ verstärkt wird oder nicht. 
	
	Konkret lassen sich die Ziele dieser Arbeit in zwei leitende exploratorische Fragestellungen einteilen. 
	Zunächst wird untersucht, ob sich unterschiedliche Verläufe in Hautleitwert- und Schreckreaktion abzeichnen, wenn beide Maße im Modell gleich behandelt werden.
	Bezugnehmend auf das differentielle Furchtlernen wird außerdem betrachtet, ob sich der Einfluss des Stimulustyps bedeutsam zwischen den abhängigen Variablen unterscheidet bzw. ob im Durchschnitt in beiden abhängigen Variablen in gleichem Maße zwischen dem CS+ und dem CS-- diskriminiert wird.
	Diese Leitfragen sind eingebettet in den explorativen Kontext dieser Arbeit, in dem ein multivariates Modell gebaut, analysiert und auf dessen Besonderheiten %hinsichtlich der Modellbildung ebenfalls 
	eingegangen wird.
	%Das Ziel ist es, ein Modell zu bauen, was die Daten der Akquisitionsphase möglichst gut und gleichzeitig möglichst sparsam erklärt. 
	

%\section{Exploratorische Fragestellungen}			\label{hypothesis}
%		\begin{enumerate}
%			\item[$H_1$:] Die Schreck- und Hautleitwertreaktionen unterscheiden sich auch bei statistischer Gleichbehandlung in ihren Verläufen signifikant voneinander. % (Interaktion \textit{Trial}$\times$\textit{Outcome})
%		\end{enumerate}
%	 
%		\begin{enumerate}
%			\item[$H_2$:] Die Verläufe von Schreck- und Hautleitwertreaktionen unterscheiden sich signifikant hinsichtlich der Diskrimination der Stimulustypen. %(Interaktion \textit{Trial}$\times$\textit{Outcome}$\times$\textit{Condition})
%		\end{enumerate}
%	Zuletzt wird die Fragestellung betrachtet, ob die Art der Trialreihenfolge differenzielle Einflüsse auf die Verläufe der abhängigen Variablen hat. 
%		\begin{enumerate}
%			\item[$H_3$:] Die Trialreihenfolge hat einen signifikant differentiellen Einfluss auf die Verläufe von Schreck- und Hautleitwertreaktionen.
%		\end{enumerate}

	 



%%%%%%%%%%%%%LEITFADEN BACHELORARBEIT
		%Aus dem theoretischen Hintergrund werden hier die empirisch untersuchbaren Fragestellungen und Hypothesen abgeleitet. Die Art der Hypothesen wird nach Untersuchungsziel charakterisiert (Zusammenhang zwischen mind. zwei Merkmalen, Unterschied zwischen mind. zwei Stichproben, Veränderung zwischen mind. zwei Messzeitpunkten). Gerichtete Hypothesen können nur aufgestellt werden, wenn dies aus dem aktuellen Forschungsstand klar ableitbar ist – in allen anderen Fällen sind zweiseitige Hypothesen zu formulieren. \\
		%
		%\textbf{Leitfragen:} Werden die Fragestellungen folgerichtig abgeleitet und nachvollziehbar begründet? Sind die Hypothesen logisch und formal einwandfrei?