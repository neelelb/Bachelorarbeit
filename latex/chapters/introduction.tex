

	Eine wesentliche Schwierigkeit in der experimentellen Forschung besteht seit jeher darin, Prozesse und Zustände, die man nicht direkt beobachten kann, greifbar zu machen. Es bedarf eines Indikators: eines Hilfsmittels, welches einen Zustand oder eine Entwicklung anzeigen und damit beobachtbar machen kann.  %lateinisch indicare „anzeigen“
	Ein Forschungsparadigma, für das sich diese Schwierigkeit in hohem Maße relevant zeigt, ist die Furchtkonditionierung. Mit ihr lassen sich unter anderem Prozesse wie das Furchtlernen untersuchen, bei dem auf Basis von vorherigen Erfahrungen negative Ereignisse antizipiert und vorsorgliche Schutz- oder Abwehrreaktionen in die Wege geleitet werden können. 
	Ein Verständnis für diese Prozesse und des ihnen zugrundeliegenden Furchtsystems ist von hohem Interesse. Zum Beispiel können fehlende Regulationsmechanismen in diesem anpassungsfähigen System mit pathologischen Mustern zusammenhängen, wie sie unter anderem bei Angststörungen festgestellt werden. Ein vollständiges Durchdringen der Mechanismen rings um die Entstehung und Aufrechterhaltung solcher Muster kann Behandlungs- und Forschungsmethoden verbessern und validieren.
	Anwendungsgebiete wie diese machen die Furchtkonditionierung zu einem wichtigen, experimentellen Paradigma.
	%If contained, fear is adaptive as it facilitates defensive responding, allowing the escape from, or avoidance of, dangerous situations, but if fear becomes exaggerated or is not appropriately regulated, it can develop into an anxiety disorder (Quinn & Fanselow, 2006)
	
	Um den Erwerb -- oder die Akquise -- von Furcht nachvollziehen zu können, werden verschiedene Indikatoren verwendet, wobei zwei physiologische Maße besonders verbreitet sind: die Hautleitwertreaktion und die Schreckreaktion. Bei erfolgreichem Furchtlernen wird typischerweise beobachtet, dass diese Reaktionen auf einen Reiz, der in der Vergangenheit ein negatives Ereignis vorhergesagt hat, höher ausfallen als auf einen, der diese Vorhersage nicht bereitstellt. Genannt wird diese Form des Furchterwerbs auch differentielles Furchtlernen.  
	Die Schreckreaktion besteht übergreifend aus einer körperlichen Kaskade an defensiven Bewegungen in Folge eines Schreckreizes, wobei das Schließen der Augen (die sogenannte Lidschlusskomponente) beim Menschen am häufigsten untersucht wird. In der Stärke der Schreckreaktion spiegelt sich unter anderem, wie angenehm oder unangenehm ein Reiz empfunden wird.
	Die Hautleitwertreaktion daneben beschreibt eine erhöhte Leitfähigkeit der Haut durch Schwitzen. Sie gilt vor allem als Maß für Erregung und reagiert besonders bei wichtigen oder neuen Stimuli. In Furchtkonditionierungsstudien wird sie ebenfalls traditionell als Indikator für Lernprozesse eingesetzt. 

	Eine immer noch aktuelle Fragestellung der Furchtkonditionierungsforschung ist, ob das Furchtlernen auf einem oder mehreren abgrenzbaren Prozessen basiert und welche Rolle das Bewusstsein über gelernte Verbindungen zwischen Reizen dabei spielt.
	Die Schreck- und die Hautleitwertreaktion werden dabei häufig als Indikatoren zweier verschiedener theoretischer Lernebenen herangezogen. Während die differentielle Schreckreaktion hauptsächlich mit einem automatischen und unbewussten Prozess in Verbindung gebracht wird, gilt eine Differenz in der Hautleitwertreaktion als Indiz eines kognitiven und bewussten Prozesses. 
	Ob diese Unterscheidung gerechtfertigt ist und inwiefern die beiden Maße tatsächlich unterscheidbare Informationen über Furchtlernprozesse bereitstellen, bleibt dabei größtenteils ungeklärt. 
	Im Kontext dieser Kontroverse hat diese Arbeit zum Ziel, die Verläufe der beiden Indikatoren in einer Furchtakquisition mittels eines multivariaten Verfahrens zu untersuchen. Durch diesen Auswertungsansatz können Limitationen herkömmlicher univariater Verfahren zum Teil überwunden und ein umfangreicher Einblick in das Zusammenspiel beider Maße gewonnen werden. 	
	Auf Basis dieses explorativen Ansatzes kann möglicherweise zu der Literatur rund um die Unterscheidung zweier konzeptueller Lernprozesse auf Reaktionsebene beigetragen werden.
	Außerdem erweitert die Arbeit die kaum bestehende Literatur, die Hautleitwert- und Schreckreaktion als physiologische Indikatoren in direkter Form miteinander vergleicht. 

	

%%%%%%%%%%%%%%%%%%%%%%%%%%%%%%%%%%%%%%%%%%%%%
%%%%%%%%%%%%%%%KAPITELÜBERBLICK%%%%%%%%%%%%%%
%%%%%%%%%%%%%%%%%%%%%%%%%%%%%%%%%%%%%%%%%%%%%

	Zu Beginn dieser Arbeit in Kapitel \ref{theorie} werden wesentliche Konzepte definiert und relevante Forschungsergebnisse eingeordnet. 
	Zunächst wird das grundlegende Paradigma der Furchtkonditionierung und ein Ausschnitt aus traditionellen und zeitgenössischen Konditionierungstheorien und -paradigmen beschrieben. 
	%Ohne Anspruch auf Vollständigkeit zu erheben, wird ein Ausschnitt aus traditionellen und zeitgenössischen Konditionierungstheorien thematisiert, um das Paradigma historisch und auch konzeptuell einzuordnen.
	Der anschließende Überblick über Indikatoren des Furchtlernens dient als Grundlage für den angestrebten Vergleich der Schreck- und Hautleitwertreaktionen, deren Methodik und Anwendung im Kontext der Furchtkonditionierung resümiert und verglichen wird.   
	%Dies ist die Grundlage für den Vergleich auf lerntheoretischer Ebene. 
	Die Fragestellung, ob Furchtlernen auf einem oder mehreren distinkten Prozessen basiert, wird anhand der sogenannten Ein-Prozess- und Zwei-Prozess-Theorien erläutert und außerdem einige der kontroversen Forschungsergebnisse diesbezüglich angesprochen. Hier spielen die Hautleitwert- und die Schreckreaktion als mögliche Indikatoren verschiedener Ebenen eine entscheidende Rolle.
	Auf diesen theoretischen Aspekten bauen die Ziele und exploratorischen Fragestellungen dieser Arbeit in Kapitel \ref{fragestellung} auf. In Kapitel \ref{methoden} gehe ich ausführlich auf die Methodik der Datenerhebung und Organisation dieser Arbeit sowie die verwendeten statistischen Analysen ein. Außerdem werden an dieser Stelle die Entscheidungskriterien für den multivariaten Modellbildungsprozess dargelegt. 
	Es folgt die Aufstellung des statistischen Modells und der Ergebnisbericht in Kapitel \ref{ergebnisse} sowie abschließend die Integration und Diskussion der Ergebnisse in Kapitel \ref{diskussion}. 



%%%%%%%%%%%%%LEITFADEN BACHELORARBEIT
		%Wichtig ist, dass potenzielle Leserinnen und Leser nach den ersten Sätzen (spätestens nach dem ersten Absatz) wissen, um was es in der Arbeit gehen wird und wieso die untersuchte Forschungsfrage von wissenschaftlichem Interesse ist .
		%im weiteren Verlauf der Einleitung dargestellt, welche Strategie man zur Beantwortung der Forschungsfrage bzw . zur Lösung des Problems gewählt hat
		
		%Diese Arbeit beginnt mit einer kurzen Einleitung zur Problemstellung (1-2 Seiten), in der das Thema psychologisch bzw. gesellschaftspolitisch eingeordnet wird, zentrale Konstrukte eingeführt werden sowie die Relevanz der Fragestellung deutlich wird. Zudem wird ein Überblick zu den nachfolgenden Kapiteln gegeben.
		%Leitfragen: Wird das Thema so eingeführt, dass auch Fachleute, die mit dem Themenbereich nicht vertraut sind, Inhalte und Relevanz der Arbeit einordnen können?